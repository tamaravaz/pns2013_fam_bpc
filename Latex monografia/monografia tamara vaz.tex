%% abtex2-modelo-trabalho-academico.tex, v-1.9.6 laurocesar
%% Copyright 2012-2016 by abnTeX2 group at http://www.abntex.net.br/ 
%%
%% This work may be distributed and/or modified under the
%% conditions of the LaTeX Project Public License, either version 1.3
%% of this license or (at your option) any later version.
%% The latest version of this license is in
%%   http://www.latex-project.org/lppl.txt
%% and version 1.3 or later is part of all distributions of LaTeX
%% version 2005/12/01 or later.
%%
%% This work has the LPPL maintenance status `maintained'.
%% 
%% The Current Maintainer of this work is the abnTeX2 team, led
%% by Lauro César Araujo. Further information are available on 
%% http://www.abntex.net.br/
%%
%% This work consists of the files abntex2-modelo-trabalho-academico.tex,
%% abntex2-modelo-include-comandos and abntex2-modelo-references.bib
%%

% ------------------------------------------------------------------------
% ------------------------------------------------------------------------
% abnTeX2: Modelo de Trabalho Academico (tese de doutorado, dissertacao de
% mestrado e trabalhos monograficos em geral) em conformidade com 
% ABNT NBR 14724:2011: Informacao e documentacao - Trabalhos academicos -
% Apresentacao
% ------------------------------------------------------------------------
% ------------------------------------------------------------------------

\documentclass[
	% -- opções da classe memoir --
	12pt,				% tamanho da fonte
	openright,			% capítulos começam em pág ímpar (insere página vazia caso preciso)
	twoside,			% para impressão em recto e verso. Oposto a oneside
	a4paper,			% tamanho do papel. 
	% -- opções da classe abntex2 --
	%chapter=TITLE,		% títulos de capítulos convertidos em letras maiúsculas
	%section=TITLE,		% títulos de seções convertidos em letras maiúsculas
	%subsection=TITLE,	% títulos de subseções convertidos em letras maiúsculas
	%subsubsection=TITLE,% títulos de subsubseções convertidos em letras maiúsculas
	% -- opções do pacote babel --
	english,			% idioma adicional para hifenização
	french,				% idioma adicional para hifenização
	spanish,			% idioma adicional para hifenização
	brazil				% o último idioma é o principal do documento
	]{abntex2}

% ---
% Pacotes básicos 
% ---
\usepackage{times}				% Usa a fonte Times New Roman		
\usepackage[T1]{fontenc}		% Selecao de codigos de fonte.
\usepackage[utf8]{inputenc}		% Codificacao do documento (conversão automática dos acentos)
\usepackage{lastpage}			% Usado pela Ficha catalográfica
\usepackage{indentfirst}		% Indenta o primeiro parágrafo de cada seção.
\usepackage{color}				% Controle das cores
\usepackage{graphicx}			% Inclusão de gráficos
\usepackage{microtype} 			% para melhorias de justificação
\usepackage{longtable}
\usepackage{booktabs}
\usepackage{graphicx}
\usepackage{multirow}
\usepackage{float}
\usepackage{tabularx}
\usepackage{pdflscape}
% ---
		
% ---
% Pacotes adicionais, usados apenas no âmbito do Modelo Canônico do abnteX2
% ---
\usepackage{lipsum}				% para geração de dummy text
% ---

% ---
% Pacotes de citações
% ---
\usepackage[brazilian,hyperpageref]{backref}	 % Paginas com as citações na bibl
\usepackage[alf]{abntex2cite}	% Citações padrão ABNT

% --- 
% CONFIGURAÇÕES DE PACOTES
% --- 

% ---
% Configurações do pacote backref
% Usado sem a opção hyperpageref de backref
\renewcommand{\backrefpagesname}{Citado na(s) página(s):~}
% Texto padrão antes do número das páginas
\renewcommand{\backref}{}
% Define os textos da citação
\renewcommand*{\backrefalt}[4]{
	\ifcase #1 %
		Nenhuma citação no texto.%
	\or
		Citado na página #2.%
	\else
		Citado #1 vezes nas páginas #2.%
	\fi}%
% ---

% ---
% Informações de dados para CAPA e FOLHA DE ROSTO
% ---
\titulo{Pobreza multidimensional e os beneficiários do BPC: uma comparação entre estratos de renda per capita}
\autor{Tamara Vaz de Moraes Santos}
\local{Brasília - DF}
\data{2017}
\orientador{Ana Carolina Pereira Zoghbi}
\instituicao{%
  Universidade de Brasília -- UnB
  \par
  Faculdade de Economia Administração e Contabilidade -- FACE
  \par
  Departamento de Economia}
\tipotrabalho{Monografia)}
% O preambulo deve conter o tipo do trabalho, o objetivo, 
% o nome da instituição e a área de concentração 
\preambulo{Monografia apresentada ao Departamento de Economia da Universidade de Brasília para obtenção do grau de Bacharel em Economia}
% ---


% ---
% Configurações de aparência do PDF final

% alterando o aspecto da cor azul
\definecolor{blue}{RGB}{41,5,195}

% informações do PDF
\makeatletter
\hypersetup{
     	%pagebackref=true,
		pdftitle={\@title}, 
		pdfauthor={\@author},
    	pdfsubject={\imprimirpreambulo},
	    pdfcreator={LaTeX with abnTeX2},
		pdfkeywords={abnt}{latex}{abntex}{abntex2}{trabalho acadêmico}, 
		colorlinks=true,       		% false: boxed links; true: colored links
    	linkcolor=blue,          	% color of internal links
    	citecolor=blue,        		% color of links to bibliography
    	filecolor=magenta,      		% color of file links
		urlcolor=blue,
		bookmarksdepth=4
}
\makeatother
% --- 

% --- 
% Espaçamentos entre linhas e parágrafos 
% --- 

% O tamanho do parágrafo é dado por:
\setlength{\parindent}{1.3cm}

% Controle do espaçamento entre um parágrafo e outro:
\setlength{\parskip}{0.2cm}  % tente também \onelineskip

% ---
% compila o indice
% ---
\makeindex
% ---

% ----
% Início do documento
% ----
\begin{document}

% Seleciona o idioma do documento (conforme pacotes do babel)
%\selectlanguage{english}
\selectlanguage{brazil}

% Retira espaço extra obsoleto entre as frases.
\frenchspacing 

% ----------------------------------------------------------
% ELEMENTOS PRÉ-TEXTUAIS
% ----------------------------------------------------------
% \pretextual

% ---
% Capa
% ---
\imprimircapa
% ---

% ---
% Folha de rosto
% (o * indica que haverá a ficha bibliográfica)
% ---
\imprimirfolhaderosto*
% ---

% ---
% Inserir a ficha bibliografica
% ---

% Isto é um exemplo de Ficha Catalográfica, ou ``Dados internacionais de
% catalogação-na-publicação''. Você pode utilizar este modelo como referência. 
% Porém, provavelmente a biblioteca da sua universidade lhe fornecerá um PDF
% com a ficha catalográfica definitiva após a defesa do trabalho. Quando estiver
% com o documento, salve-o como PDF no diretório do seu projeto e substitua todo
% o conteúdo de implementação deste arquivo pelo comando abaixo:
%
% \begin{fichacatalografica}
%     \includepdf{fig_ficha_catalografica.pdf}
% \end{fichacatalografica}

\begin{fichacatalografica}
	\sffamily
	\vspace*{\fill}					% Posição vertical
	\begin{center}					% Minipage Centralizado
	\fbox{\begin{minipage}[c][8cm]{13.5cm}		% Largura
	\small
	\imprimirautor
	%Sobrenome, Nome do autor
	
	\hspace{0.5cm} \imprimirtitulo  / \imprimirautor. --
	\imprimirlocal, \imprimirdata-
	
	\hspace{0.5cm} \pageref{LastPage} p. : il. (algumas color.) ; 30 cm.\\
	
	\hspace{0.5cm} \imprimirorientadorRotulo~\imprimirorientador\\
	
	\hspace{0.5cm}
	\parbox[t]{\textwidth}{\imprimirtipotrabalho~--~\imprimirinstituicao,
	\imprimirdata.}\\
	
	\hspace{0.5cm}
		1. Palavra-chave1.
		2. Palavra-chave2.
		2. Palavra-chave3.
		I. Orientador.
		II. Universidade xxx.
		III. Faculdade de xxx.
		IV. Título 			
	\end{minipage}}
	\end{center}
\end{fichacatalografica}
% ---



% ---
% Inserir folha de aprovação
% ---

% Isto é um exemplo de Folha de aprovação, elemento obrigatório da NBR
% 14724/2011 (seção 4.2.1.3). Você pode utilizar este modelo até a aprovação
% do trabalho. Após isso, substitua todo o conteúdo deste arquivo por uma
% imagem da página assinada pela banca com o comando abaixo:
%
% \includepdf{folhadeaprovacao_final.pdf}
%
\begin{folhadeaprovacao}

  \begin{center}
    {\ABNTEXchapterfont\large\imprimirautor}

    \vspace*{\fill}\vspace*{\fill}
    \begin{center}
      \ABNTEXchapterfont\bfseries\Large\imprimirtitulo
    \end{center}
    \vspace*{\fill}
    
    \hspace{.45\textwidth}
    \begin{minipage}{.5\textwidth}
        \imprimirpreambulo
    \end{minipage}%
    \vspace*{\fill}
   \end{center}
        
   Trabalho aprovado. \imprimirlocal, xx de julho de 2017:

   \assinatura{\textbf{Profª. Drª. \imprimirorientador} \\ Orientador} 
   \assinatura{\textbf{Prof} \\ Banca Examinadora}
   %\assinatura{\textbf{Professor} \\ Convidado 3}
   %\assinatura{\textbf{Professor} \\ Convidado 4}
      
   \begin{center}
    \vspace*{0.5cm}
    {\large\imprimirlocal}
    \par
    {\large\imprimirdata}
    \vspace*{1cm}
  \end{center}
  
\end{folhadeaprovacao}
% ---



% ---
% Agradecimentos
% ---
\begin{agradecimentos}


\end{agradecimentos}
% ---

% ---
% Epígrafe
% ---
\begin{epigrafe}
    \vspace*{\fill}
	\begin{flushright}
		\textit{``Grito aflito na rua do sossego'' \\
		(Alceu Valença)}
	\end{flushright}
\end{epigrafe}
% ---

% ---
% RESUMOS
% ---

% resumo em português
\setlength{\absparsep}{18pt} % ajusta o espaçamento dos parágrafos do resumo
\begin{resumo}
 

 \textbf{Palavras-chave}: 
\end{resumo}


% ---
% inserir lista de ilustrações
% ---
\pdfbookmark[0]{\listfigurename}{lof}
\listoffigures*
\cleardoublepage
% ---

% ---
% inserir lista de tabelas
% ---
\pdfbookmark[0]{\listtablename}{lot}
\listoftables*
\cleardoublepage
% ---

% ---
% inserir lista de abreviaturas e siglas
% ---
\begin{siglas}
  \item[ABNT] Associação Brasileira de Normas Técnicas
  \item[abnTeX] ABsurdas Normas para TeX
\end{siglas}
% ---

% ---
% inserir lista de símbolos
% ---
\begin{simbolos}
  \item[$ \Gamma $] Letra grega Gama
  \item[$ \Lambda $] Lambda
  \item[$ \zeta $] Letra grega minúscula zeta
  \item[$ \in $] Pertence
\end{simbolos}
% ---

% ---
% inserir o sumario
% ---
\pdfbookmark[0]{\contentsname}{toc}
\tableofcontents*
\cleardoublepage
% ---



% ----------------------------------------------------------
% ELEMENTOS TEXTUAIS
% ----------------------------------------------------------
\textual

% ---
% Introdução
% ---
\chapter{Introdução}
% ---

% ---
% Programa de Prestação Continuada e sua judicialização
% ---
\chapter{Programa de Prestação Continuada e sua judicialização}
O Benefício de Prestação Continuada é a garantia de um salário mínimo mensal, transferido de modo incondicional e idenpedente de qualquer contribuição prévia para o sistema de seguridade social. O benefício é destinado ao idoso com mais de 65 anos de idade e pessoa com deficiência que comprovem não possuir meios de prover a própria manutenção ou de tê-la provida pelos membros familiares residentes no mesmo domicílio. O benefício faz parte da política de assistência social, coordenado pelo Ministério do Desenvolvimento Social e Combate à Fome -- MDS, operacionalizado pelo Instituto Nacional do Seguro Social -- INSS e tem-se assegurado na constituição. Não obstante, sua regulamentação só ocorreu em 1993 na Lei Orgânica da Assistência Social (Loas), sendo implantada em 1996 após o Decreto n. 1744/1995 \cite{bpc_stf}. As regras de legebilidade foram redefinidas diversas vezes, a apresentada aqui será a mais atual, estabelecida peloXXXXXXXXXXXXXXXXXX


	\section{Regras de elegibilidade }
		\subsection{Carência de meios para a manutenção}
		A LOAS define quatro critérios para a definição dos elegíveis ao benefício: insuficiência de meios de provimento, conceito familiar, definição de idoso e  . A incapacidade de provimento mínimo é definida como a pessoa que tenha renda \textit{per capita} inferior a 1/4 do salário mínimo vigente. Ademais, nos casos de concessão do benefício para pessoas com deficiências, essas não poderão exercer atividade remunerada, excedo na condição de aprendiz por um prazo máximo de dois anos. Segundo \citeonline{adriana2016}, esse critério é largamente utilizado nos programas governamentais, de modo a facilitar a operacionalização dos programas e evitar o tratamento não isonômico. Alem disso, quando instituído o salário mínimo na constituição de 1998, sua definição presumia que o valor era capaz de atender às necessidades básicas de uma família mononuclear, ou seja: pessoa, conjugue e dois filhos, corroborando o corte de 1/4 do salário mínimo instituído.
		

		O conceito de família definido pelo programa é determinante para o cálculo da renda mínima de elegibilidade. A LOAS explicita a inclusão nesse cálculo as rendas do requerente ao benefício, seu cônjuge, filhos e enteados solteiros, irmãos solteiros, os pais e, na ausência de um deles, a madrasta ou o padrasto e menores tutelados, desde que esses integrantes familiares vivam sobre o mesmo teto. Como explicitado, o conceito familiar no BPC não é definido estritamente segundo a existência de uma unidade de consumo. \citeonline{medeiros2009mudancca} afirma que a definição atual de família do programa traz distorssoes, podendo superestimar a renda de algumas famílias pobres ou subestimar a capacidade de prover o sustento de famílias que tenham filhos e irmãos casados ou demais parentes mais ricos. Como essa definição impacta diretamente o cálculo da renda \textit{per capita}, estudos já verificaram o efeito de uma mudança nesse conceito. A mudança que considere a unidade domiciliar de consumo como família, traria a exclusão e introdução de beneficiários, tendo em média efeito nulo líquido. No entanto, traria uma maior focalização do programa \cite{medeiros2009mudancca,fambpcfreitas}.
		
		O conceito de idoso segue atualmente o descrito no Estatuto do Idoso de 2003, que considera idoso pessoas com 65 anos ou mais. A definição de deficiência segue a Convenção sobre Direitos das Pessoas com Deficiência das Nações Unidas, que estabelece que o a deficiência está em constante transformação 
		
		
		
		
		
\section{Tendência e entendimento atual das judicializações}
% ---

% ---
% Pobreza e sua dimensões
% ---
\chapter{Pobreza e sua dimensões}
O grande pressuposto das análises baseadas na dimensão de renda para a análise de pobreza é que essa é capaz de refletir a miséria do indivíduo em todas as demais dimensões. Isso corre por tratar-se por hipótese que a renda é capaz de comprar o bem estar básico. A linha de pobreza reflete, ou deveria, o mínimo necessário para a existência. Decorre imediatamente dessa suposição que existem mercados acessíveis em que esse indivíduo possa recorrer afim de "comprar" o seu bem estar. No entando, as nítidas imperfeições de mercado podem fazer com que mesmo havendo recurso não haja esse acesso aos bens básico. Outra discussão seria a homogeneidade de um corte de renda igual para todos os indivíduos: notoriamente uma simples diferença geográfica pode alterar o nível de pobreza para um mesmo corte de renda. 



% ---

% ---
% Métodos e Procedimentos
% ---
\chapter{Métodos e procedimentos}
	\section{Dados}
	A Pesquisa Nacional de Saúde de 2013 (PNS), do Instituto Brasileiro de Geografia e Estatística (IBGE), foi escolhida para prover os dados necessários para estimar o número de elegíveis idosos e pessoas com deficiência do BPC, calcular suas rendas familiares \textit{per capita} e índices de pobreza multidimensional. A escolha é pautada na abrangência nacional e disponibilidade de informações necessárias aos objetivos descritos acima, tais como: reldeação de parentesco entre membros do domicílio, recebimento  aposentadoria, presença de deficiência por tipos, grau de limitação das atividades habituais causadas pela deficiência, rendas do indivíduo, informações sobre saúde e domicílio, etc.   
	O tamanho da amostra é de aproximadamente 206 mil indivíduos, representando uma população de 200 milhões. A subpopulação de interesse são os indivíduos que são elegíveis ao recebimento do BPC, excluindo-se a regra de renda, e os indivíduos que compõem o mesmo domicílio do suposto beneficiário, somando umas população de xxxx milhões, sendo 4,2 milhões de preeleitos .
	
	\section{Identificação dos pré elegíveis e reconstrução da família BPC}
	
	A Pesquisa nacional de saúde define o conceito de família como sendo um arranjo familiar domiciliar, consistindo em um conjunto de parentes que vivem sob o mesmo teto, eventualmente sendo adicionadas pessoas que compartilhem recursos ou despesas dentro desse domicílio. As relações de parentesco na PNS são definidas entre cada componente da família e a pessoa responsável pela Unidade familiar. O responsável pelo domicílio é eleito pelo próprio morador entrevistado. Em contraste com a PNS, o BPC usa uma definição de família em que o próprio beneficiário é colocado como a pessoa de referência. Ademais, a Lei Orgânica de Assistência Social elenca os possíveis parentes que podem fazer parte dessa família do beneficiário:
	
	\begin{citacao}
		§ 1º Para os efeitos do disposto no caput, a família é composta pelo requerente, o cônjuge ou companheiro, os pais e, na ausência de um deles, a madrasta ou o padrasto, os irmãos solteiros, os filhos e enteados solteiros e os menores tutelados, desde que vivam sob o mesmo teto \cite[art.20]{loas2011}.
	\end{citacao}
	
	Pode-se assumir que o conceito de família no BPC está contido na definição da PNS. No entanto, esses conceitos apresentam diversas complicações quanto a  possibilidade de comparação fidedigna. Primeiro, os conceitos de pessoa de referência não são os mesmos, de modo que para se obter o grupo familiar do BPC a partir da família PNS, deve-se supor que o beneficiário é a pessoa de referência e reclassificar os demais. Outra dificuldade está na inexistência de representação das relações entre os componentes do domicílio, exceto com o responsável. Ou seja, embora seja possível saber que há famílias conviventes, não é trivial reconstruir com precisão os demais laços entre os indivíduos do domicílio. Além disso, a existência das categorias de outros parentes e não parentes inviabiliza qualquer suposição das relações desses com os demais. 
	
	A identificação dos beneficiários do BPC foi feita em duas etapas: a primeira foi a identificação dos preeleitos, em que há uma aplicação de filtros que reconstruíssem as regras de elegibilidade, exceto renda \textit{per capita}, dada a necessidade de captar beneficiários que estão acima do corte de elegibilidade; depois o cálculo da renda per capita familiar.
	
	A primeira regra a ser observada é o público alvo: idosos e pessoas com deficiência. Para captar os idosos, foi criada uma \textit{dummy} indicando se o indivíduo tem 65 anos ou mais e não era beneficiário no âmbito da seguridade social (como aposentadoria e pensão). No caso das pessoas com deficiência, além de não poderem ser beneficiários de aposentadorias ou pensões, o BPC explicita critérios para definir deficiência em duas etapas: uma avaliação médica e uma social, que investiga restrições provenientes da interação entre deficiência e o meio em que vive. Como esse critério é genérico e abstrato, utilizou-se de dois blocos de perguntas disponíveis na PNS: o primeiro identifica se o indivíduo tem alguma deficiência intelectual, física, auditiva e visual. O segundo refere-se ao grau de limitação das atividades habituais geradas por essas deficiências. Essas limitações são  classificadas como: não limita, um pouco, moderadamente, intensamente e muito intensamente/ não consegue. Para esse estudo foram considerados deficientes elegíveis pessoas com grau de limitação maior ou igual a moderado. A tabela \ref*{tab_resumo_regras} apresenta as principais regras de elegibilidade do programa.
	
	\begin{table}[H]
		\footnotesize
		\centering
		\caption{Regras de elegibilidade para o BPC}
		\label{tab_resumo_regras}
			\resizebox{\textwidth}{!}{
		\begin{tabular}{|m{7cm}|m{7cm}|}
			\hline
			\multicolumn{1}{|c|}{\textbf{Idoso}}                                                                                                                                                                    & \multicolumn{1}{c|}{\textbf{Pessoa com deficiência}}                                                                                                                                                    \\ \hline
			Mínimo de 65 anos                                                                                                                                                                                       & Condição incapacitante para a vida independente e para o trabalho atestada pela perícia médica e social do INSS                                                                                       \\ \hline
			Renda per capita familiar de até 1/4 de salário mínimo                                                                                                                                                  & Renda per capita familiar de até 1/4 de salário mínimo                                                                                                                                                  \\ \hline
			Não acumular com aposentadorias e pensão ou de outro regime, exceto com benefícios da assistência médica e pensões especiais de natureza indenizatória & Não acumular com aposentadorias e pensão ou de outro regime, exceto com benefícios da assistência médica, pensões especiais de natureza indenizatória e remuneração advinda de contrato de aprendizagem \\ \hline
		\end{tabular}
	}
	\end{table}
	
	
	Após a identificação das pessoas preeleitas ao benefício, foram replicados os domicílios que tivessem mais de um preeleito de modo que cada domicílio replicado tivesse um dos preeleito contido no domicílio original como pessoa de referência. Os pesos foram recalibrados para manter o total da população e domicílios. A seguir, foi indicado quem entraria em sua composição familiar para fins de cálculo de renda \textit{per capita}. O método aqui utilizado baseia-se na única informação de vínculo entre os indivíduos existente na PNS: a condição da pessoa no domicílio. Assim, foi criada uma tabela verdade que refaz as relações familiares, tomando por hipótese que a pessoa identificada como pré elegível é a pessoa de referência. Depois, são refeitas as classificações dos demais indivíduos do domicílio usando as regras descritas na LOAS para cada posição hipotética do beneficiário. O método leva em consideração a posição original do beneficiário no domicílio, a posição dos demais indivíduos, estado civil e indicativo de quem é o preeleito ao benefício. 
	
	A tabela verdade contém 360 regras, que pode ser vista no anexo \ref{anexo_reclass}, podendo-se chegar a esse número tal que:
	\begin{equation}
	((13)Pos_{titular} \cdot  (14)Pos_{todos}  \cdot (2)Estado_{civil} )-4 = 360
	\end{equation}
	
	Onde $ Pos_{titular} $ são as 13 posições passíveis de serem assumidas pelo beneficiário dentro do domicílio, $ Pos_{todos} $ são as 14 posições $ (Pos_{titular}+1) $ possíveis dos demais indivíduos do domicílio e inclusive ele mesmo, onde o adicional de uma categoria advém da possibilidade de haver outra pessoa na mesma condição que o beneficiário, e $ Estado_{civil}  $ é a possibilidade de ser solteiro ou casado. Pode-se observar que para todas as possíveis condições no domicílio, exceto pessoa de referência e conjugê, podem haver dois ou mais indivíduos na mesma posição que a do beneficiário, uma onde ele é o próprio e as demais em que a pessoa tem a mesma condição que ele. Por isso há a redução de 4 ao fim da equação, referente às duas posições que não podem existir mais de um individuo na mesma condição, tanto para solteiro quanto para não solteiro.
	
	\subsection{Ambiguidades e tratamento}
	Dentro da PNS nao há nenhuma questão que investigue as relações familiares dentro de um domicílio entre os demais componentes, exceto o responsável pelo domicílio. Assim, algumas imputações foram feitas respeitando a restrição de estado civil. As condições descritas abaixo estão em relação ao responsável do domicílio.
		\begin{itemize}
			\item Enteado é filho ou enteado do cônjuge
			\item Filhos só do responsável ou de ambos são irmão dos Enteados
			\item Enteado é irmão de enteado
			\item Irmãos são filhos ou enteados do Pai, mãe, padrasto ou madrasta
			\item Sogro(a) são casados entre si
		\end{itemize}
	No entanto, algumas das categorias apresentam pouco ou nenhum indicativo de relação de parentesco com os demais e por isso foram agrupadas como "outros", são elas: outro parente, agregados, conviventes, pensionistas, empregado doméstico e parente do empregado doméstico. Para esses casos foi considerada apenas que eles não fazem parte da família BPC do responsável pelo domicílio e nem esse faria partes daqueles. O restante das reclassificações cruzadas para essas pessoas foram marcadas como ambíguas. Outras marcações ambíguas também foram feitas quando não foi possível sequer fazer imputação. A tabela \ref*{tab_resumo_reclass} resume as regras de reclassificações e indica as ambiguidades. 
	
	
\begin{table}[H]
			\footnotesize
	\centering
	\caption{Membros da Família BPC segundo a relação do mesmo com o responsável pelo domicílio na PNS }
	\label{tab_resumo_reclass}
	\resizebox{\textwidth}{!}{%
		\begin{tabular}{@{}p{2cm}lllllllllll@{}}
			\toprule
			Condição no domicílio (PNS) & \multicolumn{11}{c}{Membros da Família BPC}                                                                                       \\ \midrule
			& Responsável & Cônjuge & Filhos/enteados & Genro/Nora & Pais & Sogro(a) & Neto(a) & Bisneto(a) & Irmão/Irmã & Avô ou avó & outros  \\
			Responsável                                 &             & sim     & sim*            & não        & sim  & não      & não     & não        & sim*       & não        & não     \\
			Cônjuge                                     & sim         &         & sim*            & não        & não  & sim      & não     & não        & não        & não        & ambiguo \\
			Filhos/enteados                             & sim         & sim     & sim*            & ambiguo    & não  & não      & ambiguo & não        & não        & não        & ambiguo \\
			Genro/Nora                                  & não         & não     & ambiguo         & não        & não  & não      & ambiguo & não        & não        & não        & ambiguo \\
			Pais                                        & sim*        & não     & não             & não        & sim  & não      & não     & não        & sim*       & não        & ambiguo \\
			Sogro(a)                                    & não         & não     & não             & não        & não  & sim      & não     & não        & não        & não        & ambiguo \\
			Neto(a)                                     & não         & não     & ambiguo         & ambiguo    & não  & não      & ambiguo & ambiguo    & não        & não        & ambiguo \\
			Bisneto(a)                                  & não         & não     & não             & não        & não  & não      & ambiguo & ambiguo    & não        & não        & ambiguo \\
			Irmão/Irmã                                  & sim*        & não     & não             & não        & sim  & não      & não     & não        & sim*       & não        & ambiguo \\
			Avô ou avó                                  & não         & não     & não             & não        & não  & não      & não     & não        & não        & ambiguo    & ambiguo \\ \bottomrule
		\end{tabular}%
	}
\end{table}
	
	\section{Cálculo da renda \textit{per capita} familiar}
    A LOAS identifica algumas fontes de renda que não podem ser computadas, são elas: renda de trabalho na posição de aprendiz ou estagiário para pessoas com deficiência, renda do BPC de um idoso no cômputo da renda de outro idoso da mesma família, rendimentos provenientes de Bolsa Família e auxílios de natureza eventual e temporária. A PNS não contém indicativo da existência de trabalho como aprendiz, e por isso foi ignorada essa regra. A variável ``outras rendas'' na PNS inclui rendimentos provenientes de juro, dividendos, programas sociais, seguro-desemprego, seguro defeso e outros rendimentos, mas não discrimina cada um e tão pouco é possível identificar o que é eventual. Assim, o cálculo da renda \textit{per capita} familiar foi feita em 4 passos: soma de todas as rendas de todos os trabalhos por grupo familiar do beneficiário, identificação dos beneficiários idosos do BPC que recebiam o benefício, verificação de beneficiários do Programa Bolsa Família e o cômputo da renda para idosos e pessoas com deficiência, separando essas famílias em rendas \textit{per capita} menor ou igual a 1/4 do salário mínimo e outra em maior que 1/4 e menor ou igual a 1/2 do salário vigente. 
      
	Para o primeiro passo, após obter o grupo familiar a partir das regras do BPC, foram computadas as rendas totais de todos os trabalhos dos integrantes da família BPC e gerada uma renda total de trabalhos familiar. No segundo, foi verificados se os preeleitos ao BPC possuíam rendimentos na variável de "outras" rendas no valor de um salário mínimo de 2013, equivalente a 678 reais. No terceiro passo foram calculados os rendimentos provenientes do Programa Bolsa Família a partir de valores típicos. Para isso, foi utilizado o método seguido por \cite{metodologiaOsorio2011} aplicando-se as regras vigentes do programa em 2013. Por fim, foram calculadas as rendas por pessoas do grupamento familiar do preeleito de modo que da variável de ``outras renda'' foram retiradas as rendas do Programa Bolsa Família. O cômputo para idoso seguiu a regra descrita na LOAS onde a o benefício BPC de um idoso em um domicílio não entra no cômputo da renda familiar do outro idoso e vice versa.
	
	
	\section{Cálculo do índice de pobreza multidimensional familiar}
	O método utilizado para criação de um índice de pobreza multidimensional familiar seguiu o proposto por \citeonline{barros2006pobreza}. Uma das principais razões para a escolha se deu pela possibilidade de cálculo a nível familiar e sua característica aditivamente agregável. Ou seja, o índice permite ser calculado também a nível de grupos, o que para o propósito desse trabalho é relevante dada a existência da necessidade de comparação das famílias dos dois grupos sinteticamente criados por cortes de renda \textit{per capita} de menos de 1/4 e maior que 1/4 e menor ou igual a 1/2 do salário minimo vigente.
	
	 Os autores usam dados provenientes da Pesquisa Nacional por Amostra de Domicílio do Brasil que não contém informações sobre saúde, e por isso a suprimiram. Essa dimensão foi incluída no presente trabalho. Outra divergência quanto as dimensões proposta pelos autores foi a supressão do indicador de escassez de recursos. Isso ocorreu em razão do uso de renda \textit{per capita} para aferir esse indicador, de modo que não foi incluído afim de não dar peso a nenhum dos estratos de renda \textit{per capita} que são nosso grupo de interesse para fins de comparação dos demais indicadores. 
	 
	 A composição do índice construído aqui inclui ao todo, 6 dimensões, 17 componentes e 33 indicadores. Os indicadores foram gerados como sendo perguntas de sim ou não, em que cada sim é considerado um aumento no nível de pobreza familiar.
	 
	  Embora seja sabido que os pesos e os indicadores devem espelhar as preferências sociais, nesse trabalho foi seguido o método mais usual de definição dos indicadores por disponibilidade da base de dados usada e pesos idênticos a todos indicadores dentro de um mesmo componente, todos componentes dentro de uma mesma dimensão e para todas as dimensões. No entanto, pode haver uma diferenciação natural dos pesos advindos das diferentes composições quantitativas das dimensões e componente. Além disso, assim como os autores, foi utilizado também o efeito cascata de indicadores, permitindo que os pesos dentro de componentes sejam alterados. 
	  
	  As sete dimensões da pobreza são: 1) vulnerabilidade, 2) acesso ao conhecimento, 3) acesso ao trabalho, 4) desenvolvimento infantil, 5) carências habitacionais e 6) saúde. A baixo foram listados os componentes que foram cada dimensão e seus indicadores. 
	  
	  \subsection{Vulnerabilidade}
	  Um dos pressupostos do BPC é de o deficiente ou idoso tem uma relação de dependência familiar, e essa deve ser responsável pela sua manutenção. Essa relação de dependência viria então a empobrecer a familiar via um aumento adicional de recursos necessários para se manterem \cite{diniz2006deficiencia}. Com esse entendimento, a dimensão de vulnerabilidade incluiu fatores que também podem contribuir com o aumento de recurso necessário para satisfazer as necessidades de uma família, são eles: fecundidade, tenção e cuidados especiais com crianças e adolescentes e dependência demográfica. Como toda família do BPC inclui idosos ou deficientes, não foi incluído nenhum indicador que captasse esse tipo de vulnerabilidade por ser redundante. A tabela \ref{ind_vulnerabilidade} apresenta os indicadores.
	  
\begin{table}[H]
	\centering
	\caption{Indicadores de vulnerabilidade familiar}
	\label{ind_vulnerabilidade}
	\resizebox{\textwidth}{!}{
	\begin{tabular}{m{6cm}ll}
		\hline
		Fecundidade                                                                                                          & A1. & Mulher grávida no domicílio                             \\
		&     &                                                         \\
		\multirow{2}{*}{\begin{tabular}[c]{@{}l@{}}Atenção e cuidados especiais com \\ crianças e adolescentes\end{tabular}} & A2. & Presença de criança                                     \\
		& A3. & Presença de criança ou adolescente                      \\
		&     &                                                         \\
		\multirow{2}{*}{Dependência demográfica}                                                                             & A4. & Ausência de cônjuge                                     \\
		& A5. & Menos da metade dos membros encontram-se em idade ativa \\ \hline
	\end{tabular}
}
\end{table}

 \subsection{Acesso insuficiente ao conhecimento}
 Sabidamente o capital humano é pilar do desenvolvimento econômico: está entre os principais meios duradouros para superar pobreza e desigualdade e proporcionar crescimento econômico . Na PNS é possível identificar o analfabetismo formal e funcional e escolaridade dos indivíduos, como mostra a tabela \ref{ind_conhecimento}.
 
 \begin{table}[H]
 	\footnotesize
 	\centering
 	\caption{Indicadores de acesso insuficiente ao conhecimento}
 	\label{ind_conhecimento}
 	\begin{tabular}{m{7cm}ll}
 		\hline
 		\multirow{2}{*}{Analfabetismo} & B1. & Presença de adulto analfabeto                   \\
 		& B2. & Presença de adulto analfabeto funcional         \\
 		&     &                                                 \\
 		\multirow{3}{*}{Escolaridade}  & B3. & Ausência de adulto com fundamental completo     \\
 		& B4. & Ausência de adulto com ensino médio completo    \\
 		& B5. & Ausência de adulto com alguma educação superior \\ \hline
 	\end{tabular}
 \end{table}

Reproduzindo o método escolhido, o efeito cascata pode ser percebido nos dois componentes dessa dimensão. No caso de analfabetismo, espera-se que analfabeto funcional seja também analfabeto formal, assim o analfabeto funcional recebe pesos duas vezes maior. Em escolaridade ocorre o mesmo: se não existe nenhum adulto com fundamental completo no domicílio, então também não existirá nenhum adulto com ensino médio completo e com educação superior, sendo o primeiro caso três vezes pior do que apenas não ter alguém ensino superior.   
 
  \subsection{Acesso ao trabalho}
  Embora os indivíduos possam não ter altos níveis de privação em relação ao acesso ao conhecimento, a pobreza pode se objetivar na privação do uso de suas capacidades dentro do mercado de trabalho. Para isso foi identificado na PNS a existência de indivíduos ocupados e a qualidade dessas ocupações, se em setor formal ou ao menos em setores fora de atividades agrícolas. Aqui também não foi investigado o rendimento dos ocupados, de modo a não dar mais peso a nenhum dos nossos estratos de rendas de interesse.  
  

	    \begin{table}[H]
	  	\footnotesize
	  	\centering
	  	\caption{Indicadores acesso ao trabalho}
	  	\label{ind_laboral}
	  	\begin{tabular}{lll}
	  		\hline
	  		Disponibilidade de trabalho                     & C1. & Menos da metade dos membros em idade ativa encontram-se ocupados \\
	  		&     &                                                                  \\
	  		\multirow{2}{*}{Qualidade do posto de trabalho} & C2. & Ausência de ocupado no setor formal                              \\
	  		& C3. & Ausência de ocupado em atividade não-agrícola                    \\ \hline
	  	\end{tabular}
	  \end{table}
  
  \subsection{Desenvolvimento infantil}
  Manter crianças dentro da escola é um esforço tanto nacional como internacional. O relatório de \citeonline{unicef2012iniciativa} demonstra que as crianças mais vulnerareis a exclusão escolar têm extensões de vulnerabilidades na vida, o que indica uma situação crítica de pobreza e subdesenvolvimento. Embora o trabalho precoce seja correlacionado com essa exclusão e um forte indicador de pobreza, esse não pode ser captado na Pesquisa nacional de saúde. Como indicador de desenvolvimento infantil foram captados, acesso à escola, progresso escolar e mortalidade infantil. A tabela \ref{ind_des_infantil} apresenta os indicadores dessa dimensão. O efeito cascata nos pesos está presente tanto no  $D1-D3$ quanto no $D4-D5$.
  
\begin{table}[H]
	\footnotesize
	\centering
	\caption{Indicadores de desenvolvimento infantil}
	\label{ind_des_infantil}
	\resizebox{\textwidth}{!}{
	\begin{tabular}{lll}
		\hline
		\multirow{3}{*}{Acesso à escola}      & D1. & Presença de ao menos uma criança de 0-6 anos fora da escola             \\
		& D2. & Presença de ao menos uma criança de 7-14 anos fora da escola            \\
		& D3. & Presença de ao menos uma criança de 7-17 anos fora da escola            \\
		&     &                                                                         \\
		\multirow{2}{*}{Progresso escolar}    & D4. & Presença de ao menos um adolescente de 10 a 14 anos analfabeto          \\
		& D5. & Presença de ao menos um jovem de 15 a 17 anos analfabeto                \\
		&     &                                                                         \\
		\multirow{2}{*}{Mortalidade infantil} & D6. & Presença de ao menos uma mãe que tenha algum filho que já tenha morrido \\
		& D7. & Presença de mãe que já teve algum filho nascido morto                   \\ \hline
	\end{tabular}
}
\end{table}

\subsection{Condições habitacionais}
  As condições habitacionais têm interseção com saúde, bem-estar e desigualdade de renda no Brasil.  \citeonline{zoghbi2007analise} demonstram que quando há um aumento de vulnerabilidade em termos habitacionais há também uma piora na autoavaliação de saúde dos indivíduos. É razoável supor que as vulnerabilidades objetivadas nas habitações não são escolhas, mas imposições por algum tipo de restrição que as impossibilite de prover melhoras. 
  
  Na PNS foi possível investigar a densidade de indivíduos por dormitórios, materiais inadequados da construção, acesso inadequado a água, disponibilidade de energia, bens duráveis e destino não adequado de lixo. Como acesso inadequado a água, entendeu-se a água advinda de de carro-pipa, armazenamentos em cisterna ou por rios, lagos e igarapés. O esgotamento sanitário inadequado são as fossas rudimentares, valas ou direto para rios, lagos ou mar. A tabela \ref{ind_habitacional} apresenta dos indicadores.
  
  
  \begin{table}[H]
  	\footnotesize
  	\centering
  	\caption{Indicadores de condições habitacionais}
  	\label{ind_habitacional}
  	\begin{tabular}{llp{9cm}}
  		\hline
  		Déficit habitacional                       & E1.  & Densidade de 2 ou mais moradores por dormitório                                                        \\
  		&      &                                                                                                        \\
  		Abrigabilidade                             & E2.  & Material de construção não é permanente                                                                \\
  		&      &                                                                                                        \\
  		Acesso a abastecimento de água             & E3.  & Acesso inadequado a água                                                                               \\
  		&      &                                                                                                        \\
  		Acesso a saneamento                        & E4.  & Esgotamento sanitário inadequado                                                                       \\
  		&      &                                                                                                        \\
  		Acesso a coleta de lixo                    & E5.  & Lixo não é coletado                                                                                    \\
  		&      &                                                                                                        \\
  		\multirow{5}{*}{Acesso a energia elétrica} & E6.  & Sem acesso a eletricidade                                                                              \\
  		& E7.  & Não tem ao menos a um dos itens: fogão ou geladeira                                                    \\
  		& E8.  & Não tem ao menos a um dos itens: fogão, geladeira, televisão                                           \\
  		& E9.  & Não tem ao menos a um dos itens: fogão, geladeira, televisão, telefone (fixo ou celular)               \\
  		& E10. & Não tem ao menos a um dos itens: fogão, geladeira, televisão, telefone (fixo ou celular) ou computador \\ \hline
  	\end{tabular}
  \end{table}

\subsection{Saúde}
A única dimensão não contida no método proposto por \citeonline{barros2006pobreza} é a de saúde por limitação da base de dados usada. Estudos de desigualdade social na saúde foram usados para definir os indicadores que comporiam essa dimensão. Nessa área, as variáveis de presença de doenças crônicas e auto avaliação do estado de saúde se mostram relevantes, em específico a autoavaliação de saúde é uma boa \textit{proxy} do real estado de saúde dos indivíduos \cite{humphries2000income,diaz2003desigualdades}. Além desses dois indicadores, a ausência de indivíduos na família com plano de saúde também foi incluída como \textit{proxy} de seguro saúde, como sugerido por \citeonline{neri2002desigualdade}.

\begin{table}[H]
	\footnotesize
	\centering
	\caption{Indicadores de saúde}
	\label{ind_saude}
	\begin{tabular}{p{3.3cm}ll}
		\hline
		Seguro saúde                     & F1. & Ausência de moradores com plano de saúde                               \\
		&     &                                                                        \\
		\multirow{2}{*}{Estado de saúde} & F2. & Presença de ao menos um morador com doença crônica ou de longa duração \\
		& F3. & Presença de adulto que classifica sua saúde como ruim ou muito ruim   \\ \hline
	\end{tabular}
\end{table}
 
 \subsection{Agregação dos indicadores e dimensões}
 
 A agregação dos indicadores foi feita de modo que primeiro temos um indicador sintético familiar de pobreza e depois podemos obter os indicadores para qualquer grupo. Assim, temos que:
 
 \begin{equation}
 S_\alpha = (\sum_{k=1}^{m} w_k B^{\alpha}_k)^{1/\alpha}
 \end{equation}
 
 onde $S$ é o indicador agregado de pobreza familiar, $\{B_k: k=1,...,m\}$ é o conjunto de indicadores a serem agregados e $w_k$ é o peso dado ao indicador elementar. Como dito anteriormente, a definição dos pesos seguiu a forma mais usada na literatura, onde são tratados de forma simétrica por falta de informação das preferências sociais. O cálculo do indicador sintético de pobreza para cada família torna-se então:
 
 \begin{equation}
 	S=\frac{1}{6} \sum_{k=1}^{6} \bigg( \frac{1}{m_k}\sum_{j=1}^{m_k} \bigg( \frac{1}{n_{jk}} \sum_{i=1}^{n_{jk}} B_{ijk} \bigg) \bigg)
 \end{equation}

Ou seja, primeiro foi calculado o indicador sintético de cada componente para as seis dimensões, sendo $ B_{ijk} $ o $i$-ésimo indicador do $j$-ésimo componente da $k$-ésima dimensão e $ n_{jk} $ o número de indicadores do $j$-ésimo componente da $k$-ésima dimensão; depois calculado o indicador de cada dimensão, sendo $m_k$ o número componentes de  $k$-ésima dimensão e por fim calculou-se o indicador geral familiar que é a média aritmética das dimensões. 

Após isso, por se aditivamente decomponível, podemos calcular o indicador geral agregando as famílias por grupos, então:

\begin{equation}
	G=\frac{1}{F_g} \sum_{i=1}^{F_g} S({f_g}_i)
\end{equation}

Onde $ G $ é a pobreza do grupo que se quer, $ F_g $ é o total de famílias nesse grupo, $ \{{f_g}_i: i=1,...,F_g \} $ é o conjunto de famílias dentro do grupo e $ S({f_g}_i) $ é o indicador geral de pobreza calculado para cada família do grupo de interesse. 
% ---

% ---
% Resultados
% ---
\chapter{Resultados}
% ---
\section{Análise Descritiva}
Em 2013 no Brasil, segundo a Pesquisa Nacional de Saúde (PNS), havia 17,9 milhões de pessoas com 65 anos ou mais (8,9\%) e 14,7 milhões de pessoas que diziam ter alguma deficiência intelectual, visual, física ou auditiva (7,3\%) com qualquer grau de limitação advinda dessas deficiências. A distribuição percentual destas populações por condição no domicílio, é apresentada na tabela \ref*{tab_prop_byc004}.

A categoria de pessoa responsável pelo domicílio agrega a maioria das pessoas, no entanto há uma diferença entre idosos e deficientes. Os idosos estão concentrados majoritariamente em três categorias: pessoa responsável pelo domicílio, cônjuge ou companheiro e pai, mãe, padrasto ou madrasta; 61\%, 22\% e 10\%, respectivamente. As pessoas com deficiência estão em mais categorias, concentrando-se, como mais de 90\%, em 4 categorias: responsável pelo domicílio, cônjuge, filhos e pai, pai, mãe, padrasto ou madrasta; 46\%, 22\%, 10,5\% e 4\%, respectivamente.

\begin{table}[H]
	\footnotesize
	\caption{Brasil -- Distribuição percentual da população de 65 anos ou mais e pessoas com alguma deficiência, segundo a condição no domicílio, 2013}
	\label{tab_prop_byc004}
	\begin{tabular}{@{}lm{4cm}m{3cm}@{}}
		\toprule
		\textbf{Condição no domicílio}                         & \textbf{Pessoa com Deficiência} & \textbf{Idoso}  \\ \midrule
		Pessoa responsável pelo domicílio                      & 46,49                           & 61,49           \\
		Cônjuge ou companheiro(a) de sexo diferente            & 21,98                           & 22,03           \\
		Cônjuge ou companheiro(a) do mesmo sexo                & 0,02                            & 0               \\
		Filho(a) do responsável e do cônjuge                   & 10,51                           & 0,04            \\
		Filho(a) somente do responsável                        & 8,17                            & 0,17            \\
		Enteado(a)                                             & 0,93                            & 0               \\
		Genro ou nora                                          & 0,21                            & 0,07            \\
		Pai, mãe, padrasto ou madrasta                         & 4,34                            & 9,86            \\
		Sogro(a)                                               & 0,94                            & 2,41            \\
		Neto(a)                                                & 1,73                            & 0               \\
		Bisneto(a)                                             & 0,01                            & 0               \\
		Irmão ou irmã                                          & 2,37                            & 1,67            \\
		Avô ou avó                                             & 0,18                            & 0,64            \\
		Outro parente                                          & 1,55                            & 1,04            \\
		Agregado(a) – Não parente que não compartilha despesas & 0,19                            & 0,24            \\
		Convivente – Não parente que compartilha despesas      & 0,29                            & 0,26            \\
		Pensionista                                            & 0,05                            & 0,04            \\
		Empregado(a) doméstico(a)                              & 0,03                            & 0,02            \\ \midrule
		\textbf{Total}                                         & \textbf{100,00}                 & \textbf{100,00} \\ \bottomrule
	\end{tabular}
		\legend{\footnotesize Fonte: Elaboração própria a partir da PNS 2013}
\end{table}

Quando selecionada as pessoas preeleitas ao recebimento do benefício, ou seja, aqueles que respeitaram todas as regras de elegibilidade do BPC, exceto de renda, foi encontrado um total de 4,2 milhões de pessoas. Desse total, 2,4 milhões foram classificados como espécie BPC idoso e 1,8 milhões de BPC deficiente, 57\% e 43\%, respectivamente. A distribuição percentual destas populações, por condição no domicílio e espécie do benefício, é apresentada na tabela \ref*{tab_prop_byc004_preeleito}


Quase 90\% do total de preeleitos estão na condições de responsável pelo domicílio, cônjuge, filhos ou pai, mãe, padrasto ou madrasta. Esse dado é relevante: essas posições são as que apresentam maior nível de acurácia na determinação do grupamento familiar BPC do beneficiário. A posição de menor nível de garantia de identificação do grupamento familiar estão em "outros" e representam  menos de 3\% do total. Subdividindo-se por espécie do benefício (se destinada à pessoa com deficiência ou ao idoso), a categoria de idoso apresenta ainda melhor situação para a reclassificação: mais de 90\% estão em três categorias e menos de 2\% estao em ``outro''. A despeito de as pessoas com deficiência estarem em mais categorias, ainda predominam-se, com 85\%, nas categorias de melhor nível de classificação: responsável pelo domicílio, cônjuge e filhos.
 
\begin{table}[H]
	\footnotesize
	\centering
	\caption{Brasil -- Distribuição percentual da população preeleita segundo a condição no domicílio e espécie do benefício, 2013}
	\label{tab_prop_byc004_preeleito}
	\begin{tabular}{@{}llp{3cm}p{3cm}@{}}
		\toprule
		\textbf{Condição no domicílio}       & \textbf{Pessoa com Deficiência} & \textbf{Idoso} & \textbf{Total} \\ \midrule
		Pessoa responsável pelo domicílio    & 21,85                           & 46,1           & 35,95          \\
		Cônjuge                              & 14,8                            & 38,27          & 28,45          \\
		Filho(a) do responsável e do cônjuge & 25,59                           & 0,01           & 10,71          \\
		Filho(a) somente do responsável      & 20,41                           & 0,23           & 8,67           \\
		Enteado(a)                           & 2,95                            & 0              & 1,23           \\
		Genro ou nora                        & 0,03                            & 0,11           & 0,07           \\
		Pai, mãe, padrasto ou madrasta       & 0,64                            & 9,06           & 5,54           \\
		Sogro(a)                             & 0,16                            & 2,07           & 1,27           \\
		Neto(a)                              & 5,52                            & 0              & 2,31           \\
		Bisneto(a)                           & 0,01                            & 0              & 0              \\
		Irmão ou irmã                        & 4,09                            & 1,73           & 2,71           \\
		Avô ou avó                           & 0                               & 0,76           & 0,44           \\
		Outros                               & 3,96                            & 1,66           & 2,62           \\ \midrule
		\textbf{Total}                       & \textbf{100}                    & \textbf{100}   & \textbf{100}   \\ \bottomrule
	\end{tabular}
\end{table}

A soma total de indivíduos passíveis de serem reclassificados como grupo familiar do BPC é de 6,4 milhões. A taxa de reclassificação total foi de 98,5\%. Quando subdividido por espécie do benefício, as famílias com o beneficiário idoso atingiu 99,4\% de reclassificação e as pessoas com alguma deficiência com restrição moderada tiveram 96,2\%. A menor taxa de sucesso nas reclassificações para benefício ao deficiente era esperado, dada a maior ocorrência de beneficiários em posições de menor acurácia na reconstrução de seus laços familiares e, por consequência, maior ocorrência de ambiguidades. A tabela \ref*{tab_per_reclas} apresenta o resultado. 

\begin{table}[H]
	\footnotesize
	\centering
	\caption{Brasil -- Percentual de reclassificações por tipo, 2013}
	\label{tab_per_reclas}
	\begin{tabular}{@{}p{4.5cm}p{4cm}p{3cm}p{3cm}@{}}
		\toprule
		\textbf{Situação da reclassificação}     & \textbf{Pessoa com Deficiência} & \textbf{Idoso} & \textbf{Total} \\ \midrule
		Reclassificado                           & 96,22                  & 99,44 & 98,55 \\                 
		Não reclassificado                       & 3,78                   & 0,56  & 1,45  \\ \midrule
		Total                                    & 100                    & 100   & 100   \\ \midrule
		\multicolumn{4}{c}{Reclassificado} \\ \midrule
		\multicolumn{1}{r}{Entra na família}     & 91,11                  & 95,45 & 94,17 \\               
		\multicolumn{1}{r}{Não entra na família} & 5,11                   & 4,00  & 4,38  \\ \midrule  
		Total                                    & 100                    & 100   & 100   \\ \bottomrule    
	\end{tabular}
\end{table}

Excluídas as pessoas que não entram na família BPC e os indivíduos com reclassificação ambígua, a média de pessoas por família é de 1,7 pessoas. O próprio beneficiário representa 65\% do total seguido de cônjuge com 18\%. Para o benefício destinado à pessoas com deficiência, essa estatística se modifica, sendo majoritariamente formada pelo próprio beneficiário e seus filhos. A distribuição da população reclassificada de acordo com a classificação do grupo familiar BPC está na tabela \ref*{tab_reclass_cond}.

\begin{table}[H]
	\footnotesize
	\centering
	\caption{Brasil -- Distribuição percentual da população reclassificada por condição da família BPC, 2013}
	\label{tab_reclass_cond}
	\begin{tabular}{@{}p{4.5cm}p{4cm}p{3cm}p{3cm}@{}}
		\toprule
		\textbf{Classificação família BPC} & \textbf{Pessoa com Deficiência} & \textbf{Idoso} & \textbf{Total} \\ \midrule
		Beneficiário                       & 67,07                           & 63,72          & 65,08          \\
		Cônjuge/companheiro(a)             & 4,94                            & 28,18          & 18,81          \\
		Filhos                             & 14,74                           & 2,45           & 2,10           \\
		Pai/mãe/madrasta/padrasto          & 2,75                            & 0,55           & 6,31           \\
		Ambíguo                            & 3,78                            & 0,56           & 1,45           \\
		Irmãos                             & 1,59                            & 0,55           & 1,86           \\
		Enteados                           & 0,03                            & 0,00           & 0,01           \\
		Não entra                          & 5,11                            & 4,00           & 4,38           \\ \midrule
		Total                              & 100,00                          & 100,00         & 100,00         \\ \bottomrule
	\end{tabular}
\end{table}

Como mostra a tabela \ref{tab_eleitos_renda}, 72,9\% dos preeleitos se enquadram na regra de até 1/4 de salário mínimo \textit{per capita} e apenas 6,6\% estão em até 1/2. Proporcionalmente, as pessoas com deficiência se adequaram menos a regra vigente de inclusão no BPC. Um motivo possível pode ser a forma diferenciada do cálculo para essas pessoas, as quais podem ter incluídos o BPC de outro membro na família no cálculo de sua renda de debilidade. 

\begin{table}[H]
	\footnotesize
	\centering
	\caption{Proporção de preeleitos por estrato de renda per capita}
	\label{tab_eleitos_renda}
	\begin{tabular}{@{}p{11cm}lll@{}}
		\toprule
		\textbf{Renda per capita} & \textbf{Total} & \textbf{Idoso} & \textbf{Deficiência} \\ \midrule
		Menor que 1/4    & 72,9  & 68,5  & 79,5        \\
		Entre 1/4 e 1/2  & 6,6   & 5,0   & 9,0         \\
		Maior que 1/2    & 20,6  & 26,6  & 11,5        \\ \midrule
		Total            & 100   & 100   & 100         \\ \bottomrule
	\end{tabular}
\end{table}

As famílias que se enquadram na atual regra de 1/4 de renda \textit{per capita} têm predominantemente renda zero, 45\% e 49\%  para pessoas com deficiência e idosos, respectivamente. Em relação as famílias que estão no estrato de renda maior do que a regra atual, mais de 30\% estão no limite de 1/2 do salário mínimo. A figura \ref{rend_resumo_figura} mostra a proporção de familiar por renda \textit{per capita}.

\begin{center}
	\includegraphics[]{distrbuicao_percapita.png}
	\begin{figure}[!h]
		\caption{Proporção de famílias por renda per capita }
		\label{rend_resumo_figura}
	\end{figure}
\end{center}

\section{Pobreza em grupos de estrato de renda per capita do BPC}

Uma análise individual dos indicadores de cada componente das seis dimensões não demonstra uma pior situação geral para nenhum dos estrados de rendas de interesse: há uma alternância de maior vulnerabilidade por grupos para cada indicador.  As questões mais relevantes para ambos grupos, que indicam vulnerabilidade,  são a ausência de cônjuge, baixa proporção de indivíduos em idade ativa,  baixo acesso ao conhecimento, má situação no acesso ao trabalho, não possuir computador, falta de planos de saúde e existência de membros com doenças crônicas. A baixa vulnerabilidade nos indicadores de desenvolvimento infantil, se deve a baixa ocorrência de crianças no domicílio.  A figura \ref{ind_resumo_figura} apresenta as proporções de famílias com vulnerabilidades nos indicadores apresentados, esses estão ordenador por dimensões. 

\begin{landscape}
\begin{center}
	\includegraphics[width= 23cm, height=14.5cm ]{resumo_indicadores.png}
	\begin{figure}[!h]
		\caption{Proporção de famílias com vulnerabilidades nos indicadores de pobreza}
		\label{ind_resumo_figura}
	\end{figure}
\end{center}
\end{landscape}

Como mostra a tabela \ref{tab_graupobreza}, o grau de pobreza entre os estratos de renda \textit{per capita} são idênticos, ambos têm 40\% de pobreza. No entanto, há dimensões em que o um grupo pode ser considerado mais pobre e outro em que inverte-se a pobreza entre os grupos, embora essa diferença não ultrapasse os 7 ponto, exceto no caso de acesso ao trabalho, em que o estrato de renda \textit{per capita} abaixo de 1/4 tem uma condição de 15 ponto pior do que o outro estrato. 
 
 Mesmo considerando separadamente benefícios ao idoso e ao deficiente e por estrato de renda, a diferença entre o indicador geral não é tão relevante, variando em no máximo dois pontos. A maior diferença encontrada está entre os deficientes, sendo que os com renda maior que um quarto do salário mínimo estão em situação um pouco melhor do que os com renda menor que um quarto. Comparando todas as dimensões aferidas da pobreza, os dois grupos continuam similares, havendo alternância do estrato em pior situação nas dimensões. 

\begin{table}[H]
	\footnotesize
	\centering
	\caption{Grau multidimensional de pobreza -- estrato de renda per capita}
	\label{tab_graupobreza}
	\begin{tabular}{@{}m{5cm}lllllll@{}}
		\toprule
		\multirow{2}{*}{Dimensão} & \multirow{2}{*}{Total} & \multicolumn{2}{c}{Total}                                            & \multicolumn{2}{c}{Idoso}                                            & \multicolumn{2}{c}{Deficiente}                                       \\ \cmidrule(l){3-8} 
		&                        & Até 1/4 & \begin{tabular}[c]{@{}l@{}}Entre 1/4\\  e 1/2\end{tabular} & Até 1/4 & \begin{tabular}[c]{@{}l@{}}Entre 1/4\\  e 1/2\end{tabular} & Até 1/4 & \begin{tabular}[c]{@{}l@{}}Entre 1/4 \\ e 1/2\end{tabular} \\ \midrule
		Indicador Geral           & 40,3                   & 40,3    & 40,2                                                       & 40,7    & 42,5                                          & 39,8    & 37,8                                                       \\
		Vulnerabilidade           & 23,9                   & 24,2    & 19,1                                                       & 24,8    & 22,7                                          & 23,5    & 15,3                                                       \\
		Acesso ao conhecimento    & 53,5                   & 53,1    & 59,6                                                       & 60,9    & 68,5                                          & 44,3    & 50,4                                                       \\
		Acesso ao trabalho        & 98,1                   & 99,1    & 84,0                                                       & 98,9    & 88,3                                          & 99,4    & 79,6                                                       \\
		Desenvolvimento infantil  & 1,4                    & 1,3     & 2,0                                                        & 0,0     & 0,1                                           & 2,7     & 4,0                                                        \\
		Condições habitacionais   & 12,0                   & 11,5    & 18,6                                                       & 10,0    & 17,2                                          & 13,3    & 20,0                                                       \\
		Saúde                     & 52,8                   & 52,4    & 57,9                                                       & 49,4    & 58,2                                          & 55,8    & 57,5                                                       \\ \bottomrule
	\end{tabular}
\end{table}

Como os indicadores por grupos são médios, poderia haver uma diferença quanto ao total da população acumulada em níveis de pobreza. No entanto, mesmo essa comparação demostra uma similaridade entre estratos e espécie do benefício: 50\% das famílias mais pobres têm índices de pobreza em torno de 40\%, como mostra a figura \ref{ind_ac_figura}.

\begin{center}
	\includegraphics[scale=0.93]{prop_ac_indicegeral_especie.png}
	\begin{figure}[!h]
		\caption{Distribuiçã acumulada das famílias de acordo com o indicador de pobreza familiar}
		\label{ind_ac_figura}
	\end{figure}
\end{center}

% ---
% Conclusão
% ---
\chapter{Conclusão}

Em 2013 o Supremo Tribunal Federal (STF) julgou que, para o deferimento do
Benefício de prestação continuada, podem ser consideradas outros indicadores de miserabilidade além das regras existentes. Os demais critérios que comprovem a situação de miserabilidade do grupo familiar e da situação de vulnerabilidade carecem de regulamento e estão omitidos na Lei Orgânica de Assistência Social. Na pratica, o judiciário tem deferido o pedido do benefício para rendas \textit{per capita} de até meio salário mínimo. Dessa forma, o critério de pobreza para a concessão do benefício se tornou subjetivo, ficando à mercê da análise individual do judiciário.

O objetivo desse trabalho foi verificar se a pobreza entre os elegíveis ao BPC pela regra da LOAS e a adotada pelo judiciário são similares ou divergentes. Para isso, foi calculado um índice de pobreza multidimensional que levou em considerações dimensões comumente utilizadas nos estudos sobre o tema e comparado os dois estratos. 

Os resultados demostram que a pobreza medida pelo índice multidimensional dos dois grupos analisados não são diferentes. Além do mais, em algumas dimensões da pobreza, o estrato de renda superior a 1/4 está em pior situação do que o grupo naturalmente eleito no BPC. Se forem consideradas apenas o grau de miserabilidade, o resultados corroboram o entendimento do Supremo Tribunal Federal de que talvez os indivíduos com renda entre 1/4 e 1/2 devam ser incluídos no programa por entenderem que a pobreza entre eles não é tão diferente, mesmo com o dobro de renda por membro familiar. Outro ponto a ser destacado nos resultados é a também similaridade de pobreza entre grupos familiares de idosos e pessoas com deficiência. Atualmente, as regras do BPC tratam o cálculo da renda \textit{per capita} dos dois grupos de forma desigual, beneficiando os idosos ao nao inserir no calculo da renda familiar o rendimento de outro beneficiário idoso no mesmo domicílio. 


% ---



% ----------------------------------------------------------
% ELEMENTOS PÓS-TEXTUAIS
% ----------------------------------------------------------
\postextual
% ----------------------------------------------------------

% ----------------------------------------------------------
% Referências bibliográficas
% ----------------------------------------------------------
\bibliography{bibliografia}

% ---
% Inicia os anexos
% ---
\begin{anexosenv}
	
	% Imprime uma página indicando o início dos anexos
	\partanexos
	
	% ---
	\chapter{Tabela verdade de recriação do grupamento familiar do BPC}
	\label{anexo_reclass}
	\footnotesize
	\begin{longtable}{@{}lcclc@{}}
			\toprule
			Condição do preeleito no domicilio        & Preeleito & Solteiro & Condição no domicílio (todos)                & Reclassificação \\ \midrule
			\endfirsthead
			\multicolumn{5}{c}%
			{\tablename\ \thetable\ -- \textit{Continuação da tabela}} \\
			\toprule
		Condição do preeleito no domicilio        & Preeleito & Solteiro & Condição no domicílio (todos)                 & Reclassificação \\ \midrule
			\endhead
			\hline \multicolumn{5}{r}{\textit{Continua na próxima página}} \\
			\endfoot
			\hline
			\endlastfoot
Pessoa responsável           & 1         & 1        & Pessoa responsável           & O Proprio       \\
Pessoa responsável           & 0         & 1        & Conjuge                      & Erro            \\
Pessoa responsável           & 0         & 1        & Filho(a) responsável/cônjuge & Filho           \\
Pessoa responsável           & 0         & 1        & Filho(a) responsável         & Filho           \\
Pessoa responsável           & 0         & 1        & Enteado(a)                   & Enteado         \\
Pessoa responsável           & 0         & 1        & Genro ou nora                & Erro            \\
Pessoa responsável           & 0         & 1        & Pai, mãe, padrasto/madrasta  & Pai/Mãe         \\
Pessoa responsável           & 0         & 1        & Sogro(a)                     & Não Entra       \\
Pessoa responsável           & 0         & 1        & Neto(a)                      & Não Entra       \\
Pessoa responsável           & 0         & 1        & Bisneto(a)                   & Não Entra       \\
Pessoa responsável           & 0         & 1        & Irmão ou irmã                & Irmao/Irma      \\
Pessoa responsável           & 0         & 1        & Avô ou avó                   & Não Entra       \\
Pessoa responsável           & 0         & 1        & Outros                       & Não Entra       \\
Conjuge                      & 0         & 1        & Pessoa responsável           & Erro            \\
Conjuge                      & 1         & 1        & Conjuge                      & Erro            \\
Conjuge                      & 0         & 1        & Filho(a) responsável/cônjuge & Filho           \\
Conjuge                      & 0         & 1        & Filho(a) responsável         & Enteado         \\
Conjuge                      & 0         & 1        & Enteado(a)                   & Filho           \\
Conjuge                      & 0         & 1        & Genro ou nora                & Erro            \\
Conjuge                      & 0         & 1        & Pai, mãe, padrasto/madrasta  & Não Entra       \\
Conjuge                      & 0         & 1        & Sogro(a)                     & Pai/Mãe         \\
Conjuge                      & 0         & 1        & Neto(a)                      & Não Entra       \\
Conjuge                      & 0         & 1        & Bisneto(a)                   & Não Entra       \\
Conjuge                      & 0         & 1        & Irmão ou irmã                & Não Entra       \\
Conjuge                      & 0         & 1        & Avô ou avó                   & Não Entra       \\
Conjuge                      & 0         & 1        & Outros                       & Ambiguo         \\
Filho(a) responsável/cônjuge & 1         & 1        & Filho(a) responsável/cônjuge & O Proprio       \\
Filho(a) responsável/cônjuge & 0         & 1        & Pessoa responsável           & Pai/Mãe         \\
Filho(a) responsável/cônjuge & 0         & 1        & Conjuge                      & Pai/Mãe         \\
Filho(a) responsável/cônjuge & 0         & 1        & Filho(a) responsável/cônjuge & Irmao/Irma      \\
Filho(a) responsável/cônjuge & 0         & 1        & Filho(a) responsável         & Irmao/Irma      \\
Filho(a) responsável/cônjuge & 0         & 1        & Enteado(a)                   & Irmao/Irma      \\
Filho(a) responsável/cônjuge & 0         & 1        & Genro ou nora                & Erro            \\
Filho(a) responsável/cônjuge & 0         & 1        & Pai, mãe, padrasto/madrasta  & Não Entra       \\
Filho(a) responsável/cônjuge & 0         & 1        & Sogro(a)                     & Não Entra       \\
Filho(a) responsável/cônjuge & 0         & 1        & Neto(a)                      & Ambiguo         \\
Filho(a) responsável/cônjuge & 0         & 1        & Bisneto(a)                   & Não Entra       \\
Filho(a) responsável/cônjuge & 0         & 1        & Irmão ou irmã                & Não Entra       \\
Filho(a) responsável/cônjuge & 0         & 1        & Avô ou avó                   & Não Entra       \\
Filho(a) responsável/cônjuge & 0         & 1        & Outros                       & Ambiguo         \\
Filho(a) responsável         & 1         & 1        & Filho(a) responsável         & O Proprio       \\
Filho(a) responsável         & 0         & 1        & Pessoa responsável           & Pai/Mãe         \\
Filho(a) responsável         & 0         & 1        & Conjuge                      & Pai/Mãe         \\
Filho(a) responsável         & 0         & 1        & Filho(a) responsável/cônjuge & Irmao/Irma      \\
Filho(a) responsável         & 0         & 1        & Filho(a) responsável         & Irmao/Irma      \\
Filho(a) responsável         & 0         & 1        & Enteado(a)                   & Irmao/Irma      \\
Filho(a) responsável         & 0         & 1        & Genro ou nora                & Erro            \\
Filho(a) responsável         & 0         & 1        & Pai, mãe, padrasto/madrasta  & Não Entra       \\
Filho(a) responsável         & 0         & 1        & Sogro(a)                     & Não Entra       \\
Filho(a) responsável         & 0         & 1        & Neto(a)                      & Ambiguo         \\
Filho(a) responsável         & 0         & 1        & Bisneto(a)                   & Não Entra       \\
Filho(a) responsável         & 0         & 1        & Irmão ou irmã                & Não Entra       \\
Filho(a) responsável         & 0         & 1        & Avô ou avó                   & Não Entra       \\
Filho(a) responsável         & 0         & 1        & Outros                       & Ambiguo         \\
Enteado(a)                   & 1         & 1        & Enteado(a)                   & O Proprio       \\
Enteado(a)                   & 0         & 1        & Pessoa responsável           & Pai/Mãe         \\
Enteado(a)                   & 0         & 1        & Conjuge                      & Erro            \\
Enteado(a)                   & 0         & 1        & Filho(a) responsável/cônjuge & Irmao/Irma      \\
Enteado(a)                   & 0         & 1        & Filho(a) responsável         & Irmao/Irma      \\
Enteado(a)                   & 0         & 1        & Enteado(a)                   & Irmao/Irma      \\
Enteado(a)                   & 0         & 1        & Genro ou nora                & Erro            \\
Enteado(a)                   & 0         & 1        & Pai, mãe, padrasto/madrasta  & Não Entra       \\
Enteado(a)                   & 0         & 1        & Sogro(a)                     & Não Entra       \\
Enteado(a)                   & 0         & 1        & Neto(a)                      & Ambiguo         \\
Enteado(a)                   & 0         & 1        & Bisneto(a)                   & Não Entra       \\
Enteado(a)                   & 0         & 1        & Irmão ou irmã                & Não Entra       \\
Enteado(a)                   & 0         & 1        & Avô ou avó                   & Não Entra       \\
Enteado(a)                   & 0         & 1        & Outros                       & Ambiguo         \\
Genro ou nora                & 1         & 1        & Genro ou nora                & Erro            \\
Genro ou nora                & 0         & 1        & Pessoa responsável           & Não Entra       \\
Genro ou nora                & 0         & 1        & Conjuge                      & Erro            \\
Genro ou nora                & 0         & 1        & Filho(a) responsável/cônjuge & Não Entra       \\
Genro ou nora                & 0         & 1        & Filho(a) responsável         & Não Entra       \\
Genro ou nora                & 0         & 1        & Enteado(a)                   & Não Entra       \\
Genro ou nora                & 0         & 1        & Genro ou nora                & Erro            \\
Genro ou nora                & 0         & 1        & Pai, mãe, padrasto/madrasta  & Não Entra       \\
Genro ou nora                & 0         & 1        & Sogro(a)                     & Não Entra       \\
Genro ou nora                & 0         & 1        & Neto(a)                      & Ambiguo         \\
Genro ou nora                & 0         & 1        & Bisneto(a)                   & Não Entra       \\
Genro ou nora                & 0         & 1        & Irmão ou irmã                & Não Entra       \\
Genro ou nora                & 0         & 1        & Avô ou avó                   & Não Entra       \\
Genro ou nora                & 0         & 1        & Outros                       & Ambiguo         \\
Pai, mãe, padrasto/madrasta  & 1         & 1        & Pai, mãe, padrasto/madrasta  & O Proprio       \\
Pai, mãe, padrasto/madrasta  & 0         & 1        & Pessoa responsável           & Filho           \\
Pai, mãe, padrasto/madrasta  & 0         & 1        & Conjuge                      & Erro            \\
Pai, mãe, padrasto/madrasta  & 0         & 1        & Filho(a) responsável/cônjuge & Não Entra       \\
Pai, mãe, padrasto/madrasta  & 0         & 1        & Filho(a) responsável         & Não Entra       \\
Pai, mãe, padrasto/madrasta  & 0         & 1        & Enteado(a)                   & Não Entra       \\
Pai, mãe, padrasto/madrasta  & 0         & 1        & Genro ou nora                & Não Entra       \\
Pai, mãe, padrasto/madrasta  & 0         & 1        & Pai, mãe, padrasto/madrasta  & Não Entra       \\
Pai, mãe, padrasto/madrasta  & 0         & 1        & Sogro(a)                     & Não Entra       \\
Pai, mãe, padrasto/madrasta  & 0         & 1        & Neto(a)                      & Não Entra       \\
Pai, mãe, padrasto/madrasta  & 0         & 1        & Bisneto(a)                   & Não Entra       \\
Pai, mãe, padrasto/madrasta  & 0         & 1        & Irmão ou irmã                & Filho           \\
Pai, mãe, padrasto/madrasta  & 0         & 1        & Avô ou avó                   & Pai/Mãe         \\
Pai, mãe, padrasto/madrasta  & 0         & 1        & Outros                       & Ambiguo         \\
Sogro(a)                     & 1         & 1        & Sogro(a)                     & O Proprio       \\
Sogro(a)                     & 0         & 1        & Pessoa responsável           & Não Entra       \\
Sogro(a)                     & 0         & 1        & Conjuge                      & Erro            \\
Sogro(a)                     & 0         & 1        & Filho(a) responsável/cônjuge & Não Entra       \\
Sogro(a)                     & 0         & 1        & Filho(a) responsável         & Não Entra       \\
Sogro(a)                     & 0         & 1        & Enteado(a)                   & Não Entra       \\
Sogro(a)                     & 0         & 1        & Genro ou nora                & Não Entra       \\
Sogro(a)                     & 0         & 1        & Pai, mãe, padrasto/madrasta  & Não Entra       \\
Sogro(a)                     & 0         & 1        & Sogro(a)                     & Não Entra       \\
Sogro(a)                     & 0         & 1        & Neto(a)                      & Não Entra       \\
Sogro(a)                     & 0         & 1        & Bisneto(a)                   & Não Entra       \\
Sogro(a)                     & 0         & 1        & Irmão ou irmã                & Não Entra       \\
Sogro(a)                     & 0         & 1        & Avô ou avó                   & Não Entra       \\
Sogro(a)                     & 0         & 1        & Outros                       & Ambiguo         \\
Neto(a)                      & 1         & 1        & Neto(a)                      & O Proprio       \\
Neto(a)                      & 0         & 1        & Pessoa responsável           & Não Entra       \\
Neto(a)                      & 0         & 1        & Conjuge                      & Erro            \\
Neto(a)                      & 0         & 1        & Filho(a) responsável/cônjuge & Ambiguo         \\
Neto(a)                      & 0         & 1        & Filho(a) responsável         & Ambiguo         \\
Neto(a)                      & 0         & 1        & Enteado(a)                   & Ambiguo         \\
Neto(a)                      & 0         & 1        & Genro ou nora                & Erro            \\
Neto(a)                      & 0         & 1        & Pai, mãe, padrasto/madrasta  & Não Entra       \\
Neto(a)                      & 0         & 1        & Sogro(a)                     & Não Entra       \\
Neto(a)                      & 0         & 1        & Neto(a)                      & Ambiguo         \\
Neto(a)                      & 0         & 1        & Bisneto(a)                   & Não Entra       \\
Neto(a)                      & 0         & 1        & Irmão ou irmã                & Não Entra       \\
Neto(a)                      & 0         & 1        & Avô ou avó                   & Não Entra       \\
Neto(a)                      & 0         & 1        & Outros                       & Ambiguo         \\
Bisneto(a)                   & 1         & 1        & Bisneto(a)                   & O Proprio       \\
Bisneto(a)                   & 0         & 1        & Pessoa responsável           & Não Entra       \\
Bisneto(a)                   & 0         & 1        & Conjuge                      & Erro            \\
Bisneto(a)                   & 0         & 1        & Filho(a) responsável/cônjuge & Não Entra       \\
Bisneto(a)                   & 0         & 1        & Filho(a) responsável         & Não Entra       \\
Bisneto(a)                   & 0         & 1        & Enteado(a)                   & Não Entra       \\
Bisneto(a)                   & 0         & 1        & Genro ou nora                & Erro            \\
Bisneto(a)                   & 0         & 1        & Pai, mãe, padrasto/madrasta  & Não Entra       \\
Bisneto(a)                   & 0         & 1        & Sogro(a)                     & Não Entra       \\
Bisneto(a)                   & 0         & 1        & Neto(a)                      & Ambiguo         \\
Bisneto(a)                   & 0         & 1        & Bisneto(a)                   & Ambiguo         \\
Bisneto(a)                   & 0         & 1        & Irmão ou irmã                & Não Entra       \\
Bisneto(a)                   & 0         & 1        & Avô ou avó                   & Não Entra       \\
Bisneto(a)                   & 0         & 1        & Outros                       & Ambiguo         \\
Irmão ou irmã                & 1         & 1        & Irmão ou irmã                & O Proprio       \\
Irmão ou irmã                & 0         & 1        & Pessoa responsável           & Irmao/Irma      \\
Irmão ou irmã                & 0         & 1        & Conjuge                      & Erro            \\
Irmão ou irmã                & 0         & 1        & Filho(a) responsável/cônjuge & Não Entra       \\
Irmão ou irmã                & 0         & 1        & Filho(a) responsável         & Não Entra       \\
Irmão ou irmã                & 0         & 1        & Enteado(a)                   & Não Entra       \\
Irmão ou irmã                & 0         & 1        & Genro ou nora                & Erro            \\
Irmão ou irmã                & 0         & 1        & Pai, mãe, padrasto/madrasta  & Pai/Mãe         \\
Irmão ou irmã                & 0         & 1        & Sogro(a)                     & Não Entra       \\
Irmão ou irmã                & 0         & 1        & Neto(a)                      & Não Entra       \\
Irmão ou irmã                & 0         & 1        & Bisneto(a)                   & Não Entra       \\
Irmão ou irmã                & 0         & 1        & Irmão ou irmã                & Irmao/Irma      \\
Irmão ou irmã                & 0         & 1        & Avô ou avó                   & Não Entra       \\
Irmão ou irmã                & 0         & 1        & Outros                       & Ambiguo         \\
Avô ou avó                   & 1         & 1        & Avô ou avó                   & O Proprio       \\
Avô ou avó                   & 0         & 1        & Pessoa responsável           & Não Entra       \\
Avô ou avó                   & 0         & 1        & Conjuge                      & Erro            \\
Avô ou avó                   & 0         & 1        & Filho(a) responsável/cônjuge & Não Entra       \\
Avô ou avó                   & 0         & 1        & Filho(a) responsável         & Não Entra       \\
Avô ou avó                   & 0         & 1        & Enteado(a)                   & Não Entra       \\
Avô ou avó                   & 0         & 1        & Genro ou nora                & Erro            \\
Avô ou avó                   & 0         & 1        & Pai, mãe, padrasto/madrasta  & Ambiguo         \\
Avô ou avó                   & 0         & 1        & Sogro(a)                     & Não Entra       \\
Avô ou avó                   & 0         & 1        & Neto(a)                      & Não Entra       \\
Avô ou avó                   & 0         & 1        & Bisneto(a)                   & Não Entra       \\
Avô ou avó                   & 0         & 1        & Irmão ou irmã                & Não Entra       \\
Avô ou avó                   & 0         & 1        & Avô ou avó                   & Não Entra       \\
Avô ou avó                   & 0         & 1        & Outros                       & Ambiguo         \\
Outros                       & 1         & 1        & Outros                       & O Proprio       \\
Outros                       & 0         & 1        & Pessoa responsável           & Não Entra       \\
Outros                       & 0         & 1        & Conjuge                      & Erro            \\
Outros                       & 0         & 1        & Filho(a) responsável/cônjuge & Ambiguo         \\
Outros                       & 0         & 1        & Filho(a) responsável         & Ambiguo         \\
Outros                       & 0         & 1        & Enteado(a)                   & Ambiguo         \\
Outros                       & 0         & 1        & Genro ou nora                & Erro            \\
Outros                       & 0         & 1        & Pai, mãe, padrasto/madrasta  & Ambiguo         \\
Outros                       & 0         & 1        & Sogro(a)                     & Ambiguo         \\
Outros                       & 0         & 1        & Neto(a)                      & Ambiguo         \\
Outros                       & 0         & 1        & Bisneto(a)                   & Ambiguo         \\
Outros                       & 0         & 1        & Irmão ou irmã                & Ambiguo         \\
Outros                       & 0         & 1        & Avô ou avó                   & Ambiguo         \\
Outros                       & 0         & 1        & Outros                       & Ambiguo         \\
Pessoa responsável           & 1         & 0        & Pessoa responsável           & O Proprio       \\
Pessoa responsável           & 0         & 0        & Conjuge                      & Conjuge         \\
Pessoa responsável           & 0         & 0        & Filho(a) responsável/cônjuge & Não Entra       \\
Pessoa responsável           & 0         & 0        & Filho(a) responsável         & Não Entra       \\
Pessoa responsável           & 0         & 0        & Enteado(a)                   & Não Entra       \\
Pessoa responsável           & 0         & 0        & Genro ou nora                & Não Entra       \\
Pessoa responsável           & 0         & 0        & Pai, mãe, padrasto/madrasta  & Pai/Mãe         \\
Pessoa responsável           & 0         & 0        & Sogro(a)                     & Não Entra       \\
Pessoa responsável           & 0         & 0        & Neto(a)                      & Não Entra       \\
Pessoa responsável           & 0         & 0        & Bisneto(a)                   & Não Entra       \\
Pessoa responsável           & 0         & 0        & Irmão ou irmã                & Não Entra       \\
Pessoa responsável           & 0         & 0        & Avô ou avó                   & Não Entra       \\
Pessoa responsável           & 0         & 0        & Outros                       & Não Entra       \\
Conjuge                      & 0         & 0        & Pessoa responsável           & Conjuge         \\
Conjuge                      & 1         & 0        & Conjuge                      & O Proprio       \\
Conjuge                      & 0         & 0        & Filho(a) responsável/cônjuge & Não Entra       \\
Conjuge                      & 0         & 0        & Filho(a) responsável         & Não Entra       \\
Conjuge                      & 0         & 0        & Enteado(a)                   & Não Entra       \\
Conjuge                      & 0         & 0        & Genro ou nora                & Não Entra       \\
Conjuge                      & 0         & 0        & Pai, mãe, padrasto/madrasta  & Não Entra       \\
Conjuge                      & 0         & 0        & Sogro(a)                     & Pai/Mãe         \\
Conjuge                      & 0         & 0        & Neto(a)                      & Não Entra       \\
Conjuge                      & 0         & 0        & Bisneto(a)                   & Não Entra       \\
Conjuge                      & 0         & 0        & Irmão ou irmã                & Não Entra       \\
Conjuge                      & 0         & 0        & Avô ou avó                   & Não Entra       \\
Conjuge                      & 0         & 0        & Outros                       & Ambiguo         \\
Filho(a) responsável/cônjuge & 1         & 0        & Filho(a) responsável/cônjuge & O Proprio       \\
Filho(a) responsável/cônjuge & 0         & 0        & Pessoa responsável           & Pai/Mãe         \\
Filho(a) responsável/cônjuge & 0         & 0        & Conjuge                      & Pai/Mãe         \\
Filho(a) responsável/cônjuge & 0         & 0        & Filho(a) responsável/cônjuge & Não Entra       \\
Filho(a) responsável/cônjuge & 0         & 0        & Filho(a) responsável         & Não Entra       \\
Filho(a) responsável/cônjuge & 0         & 0        & Enteado(a)                   & Não Entra       \\
Filho(a) responsável/cônjuge & 0         & 0        & Genro ou nora                & Ambiguo         \\
Filho(a) responsável/cônjuge & 0         & 0        & Pai, mãe, padrasto/madrasta  & Não Entra       \\
Filho(a) responsável/cônjuge & 0         & 0        & Sogro(a)                     & Não Entra       \\
Filho(a) responsável/cônjuge & 0         & 0        & Neto(a)                      & Não Entra       \\
Filho(a) responsável/cônjuge & 0         & 0        & Bisneto(a)                   & Não Entra       \\
Filho(a) responsável/cônjuge & 0         & 0        & Irmão ou irmã                & Não Entra       \\
Filho(a) responsável/cônjuge & 0         & 0        & Avô ou avó                   & Não Entra       \\
Filho(a) responsável/cônjuge & 0         & 0        & Outros                       & Ambiguo         \\
Filho(a) responsável         & 1         & 0        & Filho(a) responsável         & O Proprio       \\
Filho(a) responsável         & 0         & 0        & Pessoa responsável           & Pai/Mãe         \\
Filho(a) responsável         & 0         & 0        & Conjuge                      & Pai/Mãe         \\
Filho(a) responsável         & 0         & 0        & Filho(a) responsável/cônjuge & Não Entra       \\
Filho(a) responsável         & 0         & 0        & Filho(a) responsável         & Não Entra       \\
Filho(a) responsável         & 0         & 0        & Enteado(a)                   & Não Entra       \\
Filho(a) responsável         & 0         & 0        & Genro ou nora                & Ambiguo         \\
Filho(a) responsável         & 0         & 0        & Pai, mãe, padrasto/madrasta  & Não Entra       \\
Filho(a) responsável         & 0         & 0        & Sogro(a)                     & Não Entra       \\
Filho(a) responsável         & 0         & 0        & Neto(a)                      & Não Entra       \\
Filho(a) responsável         & 0         & 0        & Bisneto(a)                   & Não Entra       \\
Filho(a) responsável         & 0         & 0        & Irmão ou irmã                & Não Entra       \\
Filho(a) responsável         & 0         & 0        & Avô ou avó                   & Não Entra       \\
Filho(a) responsável         & 0         & 0        & Outros                       & Ambiguo         \\
Enteado(a)                   & 1         & 0        & Enteado(a)                   & O Proprio       \\
Enteado(a)                   & 0         & 0        & Pessoa responsável           & Pai/Mãe         \\
Enteado(a)                   & 0         & 0        & Conjuge                      & Ambiguo         \\
Enteado(a)                   & 0         & 0        & Filho(a) responsável/cônjuge & Não Entra       \\
Enteado(a)                   & 0         & 0        & Filho(a) responsável         & Não Entra       \\
Enteado(a)                   & 0         & 0        & Enteado(a)                   & Não Entra       \\
Enteado(a)                   & 0         & 0        & Genro ou nora                & Não Entra       \\
Enteado(a)                   & 0         & 0        & Pai, mãe, padrasto/madrasta  & Não Entra       \\
Enteado(a)                   & 0         & 0        & Sogro(a)                     & Não Entra       \\
Enteado(a)                   & 0         & 0        & Neto(a)                      & Não Entra       \\
Enteado(a)                   & 0         & 0        & Bisneto(a)                   & Não Entra       \\
Enteado(a)                   & 0         & 0        & Irmão ou irmã                & Não Entra       \\
Enteado(a)                   & 0         & 0        & Avô ou avó                   & Não Entra       \\
Enteado(a)                   & 0         & 0        & Outros                       & Ambiguo         \\
Genro ou nora                & 1         & 0        & Genro ou nora                & O Proprio       \\
Genro ou nora                & 0         & 0        & Pessoa responsável           & Não Entra       \\
Genro ou nora                & 0         & 0        & Conjuge                      & Não Entra       \\
Genro ou nora                & 0         & 0        & Filho(a) responsável/cônjuge & Ambiguo         \\
Genro ou nora                & 0         & 0        & Filho(a) responsável         & Ambiguo         \\
Genro ou nora                & 0         & 0        & Enteado(a)                   & Ambiguo         \\
Genro ou nora                & 0         & 0        & Genro ou nora                & Não Entra       \\
Genro ou nora                & 0         & 0        & Pai, mãe, padrasto/madrasta  & Não Entra       \\
Genro ou nora                & 0         & 0        & Sogro(a)                     & Não Entra       \\
Genro ou nora                & 0         & 0        & Neto(a)                      & Não Entra       \\
Genro ou nora                & 0         & 0        & Bisneto(a)                   & Não Entra       \\
Genro ou nora                & 0         & 0        & Irmão ou irmã                & Não Entra       \\
Genro ou nora                & 0         & 0        & Avô ou avó                   & Não Entra       \\
Genro ou nora                & 0         & 0        & Outros                       & Ambiguo         \\
Pai, mãe, padrasto/madrasta  & 1         & 0        & Pai, mãe, padrasto/madrasta  & O Proprio       \\
Pai, mãe, padrasto/madrasta  & 0         & 0        & Pessoa responsável           & Não Entra       \\
Pai, mãe, padrasto/madrasta  & 0         & 0        & Conjuge                      & Não Entra       \\
Pai, mãe, padrasto/madrasta  & 0         & 0        & Filho(a) responsável/cônjuge & Não Entra       \\
Pai, mãe, padrasto/madrasta  & 0         & 0        & Filho(a) responsável         & Não Entra       \\
Pai, mãe, padrasto/madrasta  & 0         & 0        & Enteado(a)                   & Não Entra       \\
Pai, mãe, padrasto/madrasta  & 0         & 0        & Genro ou nora                & Não Entra       \\
Pai, mãe, padrasto/madrasta  & 0         & 0        & Pai, mãe, padrasto/madrasta  & Conjuge         \\
Pai, mãe, padrasto/madrasta  & 0         & 0        & Sogro(a)                     & Não Entra       \\
Pai, mãe, padrasto/madrasta  & 0         & 0        & Neto(a)                      & Não Entra       \\
Pai, mãe, padrasto/madrasta  & 0         & 0        & Bisneto(a)                   & Não Entra       \\
Pai, mãe, padrasto/madrasta  & 0         & 0        & Irmão ou irmã                & Não Entra       \\
Pai, mãe, padrasto/madrasta  & 0         & 0        & Avô ou avó                   & Pai/Mãe         \\
Pai, mãe, padrasto/madrasta  & 0         & 0        & Outros                       & Ambiguo         \\
Sogro(a)                     & 1         & 0        & Sogro(a)                     & O Proprio       \\
Sogro(a)                     & 0         & 0        & Pessoa responsável           & Não Entra       \\
Sogro(a)                     & 0         & 0        & Conjuge                      & Não Entra       \\
Sogro(a)                     & 0         & 0        & Filho(a) responsável/cônjuge & Não Entra       \\
Sogro(a)                     & 0         & 0        & Filho(a) responsável         & Não Entra       \\
Sogro(a)                     & 0         & 0        & Enteado(a)                   & Não Entra       \\
Sogro(a)                     & 0         & 0        & Genro ou nora                & Não Entra       \\
Sogro(a)                     & 0         & 0        & Pai, mãe, padrasto/madrasta  & Não Entra       \\
Sogro(a)                     & 0         & 0        & Sogro(a)                     & Conjuge         \\
Sogro(a)                     & 0         & 0        & Neto(a)                      & Não Entra       \\
Sogro(a)                     & 0         & 0        & Bisneto(a)                   & Não Entra       \\
Sogro(a)                     & 0         & 0        & Irmão ou irmã                & Não Entra       \\
Sogro(a)                     & 0         & 0        & Avô ou avó                   & Não Entra       \\
Sogro(a)                     & 0         & 0        & Outros                       & Ambiguo         \\
Neto(a)                      & 1         & 0        & Neto(a)                      & O Proprio       \\
Neto(a)                      & 0         & 0        & Pessoa responsável           & Não Entra       \\
Neto(a)                      & 0         & 0        & Conjuge                      & Não Entra       \\
Neto(a)                      & 0         & 0        & Filho(a) responsável/cônjuge & Ambiguo         \\
Neto(a)                      & 0         & 0        & Filho(a) responsável         & Ambiguo         \\
Neto(a)                      & 0         & 0        & Enteado(a)                   & Ambiguo         \\
Neto(a)                      & 0         & 0        & Genro ou nora                & Ambiguo         \\
Neto(a)                      & 0         & 0        & Pai, mãe, padrasto/madrasta  & Não Entra       \\
Neto(a)                      & 0         & 0        & Sogro(a)                     & Não Entra       \\
Neto(a)                      & 0         & 0        & Neto(a)                      & Não Entra       \\
Neto(a)                      & 0         & 0        & Bisneto(a)                   & Não Entra       \\
Neto(a)                      & 0         & 0        & Irmão ou irmã                & Não Entra       \\
Neto(a)                      & 0         & 0        & Avô ou avó                   & Não Entra       \\
Neto(a)                      & 0         & 0        & Outros                       & Ambiguo         \\
Bisneto(a)                   & 1         & 0        & Bisneto(a)                   & O Proprio       \\
Bisneto(a)                   & 0         & 0        & Pessoa responsável           & Não Entra       \\
Bisneto(a)                   & 0         & 0        & Conjuge                      & Não Entra       \\
Bisneto(a)                   & 0         & 0        & Filho(a) responsável/cônjuge & Não Entra       \\
Bisneto(a)                   & 0         & 0        & Filho(a) responsável         & Não Entra       \\
Bisneto(a)                   & 0         & 0        & Enteado(a)                   & Não Entra       \\
Bisneto(a)                   & 0         & 0        & Genro ou nora                & Não Entra       \\
Bisneto(a)                   & 0         & 0        & Pai, mãe, padrasto/madrasta  & Não Entra       \\
Bisneto(a)                   & 0         & 0        & Sogro(a)                     & Não Entra       \\
Bisneto(a)                   & 0         & 0        & Neto(a)                      & Ambiguo         \\
Bisneto(a)                   & 0         & 0        & Bisneto(a)                   & Não Entra       \\
Bisneto(a)                   & 0         & 0        & Irmão ou irmã                & Não Entra       \\
Bisneto(a)                   & 0         & 0        & Avô ou avó                   & Não Entra       \\
Bisneto(a)                   & 0         & 0        & Outros                       & Ambiguo         \\
Irmão ou irmã                & 1         & 0        & Irmão ou irmã                & O Proprio       \\
Irmão ou irmã                & 0         & 0        & Pessoa responsável           & Não Entra       \\
Irmão ou irmã                & 0         & 0        & Conjuge                      & Não Entra       \\
Irmão ou irmã                & 0         & 0        & Filho(a) responsável/cônjuge & Não Entra       \\
Irmão ou irmã                & 0         & 0        & Filho(a) responsável         & Não Entra       \\
Irmão ou irmã                & 0         & 0        & Enteado(a)                   & Não Entra       \\
Irmão ou irmã                & 0         & 0        & Genro ou nora                & Não Entra       \\
Irmão ou irmã                & 0         & 0        & Pai, mãe, padrasto/madrasta  & Pai/Mãe         \\
Irmão ou irmã                & 0         & 0        & Sogro(a)                     & Não Entra       \\
Irmão ou irmã                & 0         & 0        & Neto(a)                      & Não Entra       \\
Irmão ou irmã                & 0         & 0        & Bisneto(a)                   & Não Entra       \\
Irmão ou irmã                & 0         & 0        & Irmão ou irmã                & Não Entra       \\
Irmão ou irmã                & 0         & 0        & Avô ou avó                   & Não Entra       \\
Irmão ou irmã                & 0         & 0        & Outros                       & Ambiguo         \\
Avô ou avó                   & 1         & 0        & Avô ou avó                   & O Proprio       \\
Avô ou avó                   & 0         & 0        & Pessoa responsável           & Não Entra       \\
Avô ou avó                   & 0         & 0        & Conjuge                      & Não Entra       \\
Avô ou avó                   & 0         & 0        & Filho(a) responsável/cônjuge & Não Entra       \\
Avô ou avó                   & 0         & 0        & Filho(a) responsável         & Não Entra       \\
Avô ou avó                   & 0         & 0        & Enteado(a)                   & Não Entra       \\
Avô ou avó                   & 0         & 0        & Genro ou nora                & Não Entra       \\
Avô ou avó                   & 0         & 0        & Pai, mãe, padrasto/madrasta  & Ambiguo         \\
Avô ou avó                   & 0         & 0        & Sogro(a)                     & Não Entra       \\
Avô ou avó                   & 0         & 0        & Neto(a)                      & Não Entra       \\
Avô ou avó                   & 0         & 0        & Bisneto(a)                   & Não Entra       \\
Avô ou avó                   & 0         & 0        & Irmão ou irmã                & Não Entra       \\
Avô ou avó                   & 0         & 0        & Avô ou avó                   & Ambiguo         \\
Avô ou avó                   & 0         & 0        & Outros                       & Ambiguo         \\
Outros                       & 1         & 0        & Outros                       & O Proprio       \\
Outros                       & 0         & 0        & Pessoa responsável           & Não Entra       \\
Outros                       & 0         & 0        & Conjuge                      & Ambiguo         \\
Outros                       & 0         & 0        & Filho(a) responsável/cônjuge & Ambiguo         \\
Outros                       & 0         & 0        & Filho(a) responsável         & Ambiguo         \\
Outros                       & 0         & 0        & Enteado(a)                   & Ambiguo         \\
Outros                       & 0         & 0        & Genro ou nora                & Ambiguo         \\
Outros                       & 0         & 0        & Pai, mãe, padrasto/madrasta  & Ambiguo         \\
Outros                       & 0         & 0        & Sogro(a)                     & Ambiguo         \\
Outros                       & 0         & 0        & Neto(a)                      & Ambiguo         \\
Outros                       & 0         & 0        & Bisneto(a)                   & Ambiguo         \\
Outros                       & 0         & 0        & Irmão ou irmã                & Ambiguo         \\
Outros                       & 0         & 0        & Avô ou avó                   & Ambiguo         \\
Outros                       & 0         & 0        & Outros                       & Ambiguo         \\ \bottomrule
\end{longtable}
	% ---
	
	
\end{anexosenv}


%---------------------------------------------------------------------
% INDICE REMISSIVO
%---------------------------------------------------------------------
\phantompart
\printindex
%---------------------------------------------------------------------

\end{document}
