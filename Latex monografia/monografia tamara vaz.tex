%% abtex2-modelo-trabalho-academico.tex, v-1.9.6 laurocesar
%% Copyright 2012-2016 by abnTeX2 group at http://www.abntex.net.br/ 
%%
%% This work may be distributed and/or modified under the
%% conditions of the LaTeX Project Public License, either version 1.3
%% of this license or (at your option) any later version.
%% The latest version of this license is in
%%   http://www.latex-project.org/lppl.txt
%% and version 1.3 or later is part of all distributions of LaTeX
%% version 2005/12/01 or later.
%%
%% This work has the LPPL maintenance status `maintained'.
%% 
%% The Current Maintainer of this work is the abnTeX2 team, led
%% by Lauro César Araujo. Further information are available on 
%% http://www.abntex.net.br/
%%
%% This work consists of the files abntex2-modelo-trabalho-academico.tex,
%% abntex2-modelo-include-comandos and abntex2-modelo-references.bib
%%

% ------------------------------------------------------------------------
% ------------------------------------------------------------------------
% abnTeX2: Modelo de Trabalho Academico (tese de doutorado, dissertacao de
% mestrado e trabalhos monograficos em geral) em conformidade com 
% ABNT NBR 14724:2011: Informacao e documentacao - Trabalhos academicos -
% Apresentacao
% ------------------------------------------------------------------------
% ------------------------------------------------------------------------

\documentclass[
	% -- opções da classe memoir --
	12pt,				% tamanho da fonte
	openright,			% capítulos começam em pág ímpar (insere página vazia caso preciso)
	twoside,			% para impressão em recto e verso. Oposto a oneside
	a4paper,			% tamanho do papel. 
	% -- opções da classe abntex2 --
	%chapter=TITLE,		% títulos de capítulos convertidos em letras maiúsculas
	%section=TITLE,		% títulos de seções convertidos em letras maiúsculas
	%subsection=TITLE,	% títulos de subseções convertidos em letras maiúsculas
	%subsubsection=TITLE,% títulos de subsubseções convertidos em letras maiúsculas
	% -- opções do pacote babel --
	english,			% idioma adicional para hifenização
	french,				% idioma adicional para hifenização
	spanish,			% idioma adicional para hifenização
	brazil				% o último idioma é o principal do documento
	]{abntex2}

% ---
% Pacotes básicos 
% ---
\usepackage{times}				% Usa a fonte Times New Roman		
\usepackage[T1]{fontenc}		% Selecao de codigos de fonte.
\usepackage[utf8]{inputenc}		% Codificacao do documento (conversão automática dos acentos)
\usepackage{lastpage}			% Usado pela Ficha catalográfica
\usepackage{indentfirst}		% Indenta o primeiro parágrafo de cada seção.
\usepackage{color}				% Controle das cores
\usepackage{graphicx}			% Inclusão de gráficos
\usepackage{microtype} 			% para melhorias de justificação
\usepackage{longtable}
\usepackage{booktabs}
\usepackage{graphicx}
\usepackage{multirow}
% ---
		
% ---
% Pacotes adicionais, usados apenas no âmbito do Modelo Canônico do abnteX2
% ---
\usepackage{lipsum}				% para geração de dummy text
% ---

% ---
% Pacotes de citações
% ---
\usepackage[brazilian,hyperpageref]{backref}	 % Paginas com as citações na bibl
\usepackage[alf]{abntex2cite}	% Citações padrão ABNT

% --- 
% CONFIGURAÇÕES DE PACOTES
% --- 

% ---
% Configurações do pacote backref
% Usado sem a opção hyperpageref de backref
\renewcommand{\backrefpagesname}{Citado na(s) página(s):~}
% Texto padrão antes do número das páginas
\renewcommand{\backref}{}
% Define os textos da citação
\renewcommand*{\backrefalt}[4]{
	\ifcase #1 %
		Nenhuma citação no texto.%
	\or
		Citado na página #2.%
	\else
		Citado #1 vezes nas páginas #2.%
	\fi}%
% ---

% ---
% Informações de dados para CAPA e FOLHA DE ROSTO
% ---
\titulo{Pobreza multidimensional e os beneficiários do BPC: uma comparação entre estratos de renda per capita}
\autor{Tamara Vaz de Moraes Santos}
\local{Brasília - DF}
\data{2017}
\orientador{Ana Carolina Pereira Zoghbi}
\instituicao{%
  Universidade de Brasília -- UnB
  \par
  Faculdade de Economia Administração e Contabilidade -- FACE
  \par
  Departamento de Economia}
\tipotrabalho{Monografia)}
% O preambulo deve conter o tipo do trabalho, o objetivo, 
% o nome da instituição e a área de concentração 
\preambulo{Monografia apresentada ao Departamento de Economia da Universidade de Brasília para obtenção do grau de Bacharel em Economia}
% ---


% ---
% Configurações de aparência do PDF final

% alterando o aspecto da cor azul
\definecolor{blue}{RGB}{41,5,195}

% informações do PDF
\makeatletter
\hypersetup{
     	%pagebackref=true,
		pdftitle={\@title}, 
		pdfauthor={\@author},
    	pdfsubject={\imprimirpreambulo},
	    pdfcreator={LaTeX with abnTeX2},
		pdfkeywords={abnt}{latex}{abntex}{abntex2}{trabalho acadêmico}, 
		colorlinks=true,       		% false: boxed links; true: colored links
    	linkcolor=blue,          	% color of internal links
    	citecolor=blue,        		% color of links to bibliography
    	filecolor=magenta,      		% color of file links
		urlcolor=blue,
		bookmarksdepth=4
}
\makeatother
% --- 

% --- 
% Espaçamentos entre linhas e parágrafos 
% --- 

% O tamanho do parágrafo é dado por:
\setlength{\parindent}{1.3cm}

% Controle do espaçamento entre um parágrafo e outro:
\setlength{\parskip}{0.2cm}  % tente também \onelineskip

% ---
% compila o indice
% ---
\makeindex
% ---

% ----
% Início do documento
% ----
\begin{document}

% Seleciona o idioma do documento (conforme pacotes do babel)
%\selectlanguage{english}
\selectlanguage{brazil}

% Retira espaço extra obsoleto entre as frases.
\frenchspacing 

% ----------------------------------------------------------
% ELEMENTOS PRÉ-TEXTUAIS
% ----------------------------------------------------------
% \pretextual

% ---
% Capa
% ---
\imprimircapa
% ---

% ---
% Folha de rosto
% (o * indica que haverá a ficha bibliográfica)
% ---
\imprimirfolhaderosto*
% ---

% ---
% Inserir a ficha bibliografica
% ---

% Isto é um exemplo de Ficha Catalográfica, ou ``Dados internacionais de
% catalogação-na-publicação''. Você pode utilizar este modelo como referência. 
% Porém, provavelmente a biblioteca da sua universidade lhe fornecerá um PDF
% com a ficha catalográfica definitiva após a defesa do trabalho. Quando estiver
% com o documento, salve-o como PDF no diretório do seu projeto e substitua todo
% o conteúdo de implementação deste arquivo pelo comando abaixo:
%
% \begin{fichacatalografica}
%     \includepdf{fig_ficha_catalografica.pdf}
% \end{fichacatalografica}

\begin{fichacatalografica}
	\sffamily
	\vspace*{\fill}					% Posição vertical
	\begin{center}					% Minipage Centralizado
	\fbox{\begin{minipage}[c][8cm]{13.5cm}		% Largura
	\small
	\imprimirautor
	%Sobrenome, Nome do autor
	
	\hspace{0.5cm} \imprimirtitulo  / \imprimirautor. --
	\imprimirlocal, \imprimirdata-
	
	\hspace{0.5cm} \pageref{LastPage} p. : il. (algumas color.) ; 30 cm.\\
	
	\hspace{0.5cm} \imprimirorientadorRotulo~\imprimirorientador\\
	
	\hspace{0.5cm}
	\parbox[t]{\textwidth}{\imprimirtipotrabalho~--~\imprimirinstituicao,
	\imprimirdata.}\\
	
	\hspace{0.5cm}
		1. Palavra-chave1.
		2. Palavra-chave2.
		2. Palavra-chave3.
		I. Orientador.
		II. Universidade xxx.
		III. Faculdade de xxx.
		IV. Título 			
	\end{minipage}}
	\end{center}
\end{fichacatalografica}
% ---



% ---
% Inserir folha de aprovação
% ---

% Isto é um exemplo de Folha de aprovação, elemento obrigatório da NBR
% 14724/2011 (seção 4.2.1.3). Você pode utilizar este modelo até a aprovação
% do trabalho. Após isso, substitua todo o conteúdo deste arquivo por uma
% imagem da página assinada pela banca com o comando abaixo:
%
% \includepdf{folhadeaprovacao_final.pdf}
%
\begin{folhadeaprovacao}

  \begin{center}
    {\ABNTEXchapterfont\large\imprimirautor}

    \vspace*{\fill}\vspace*{\fill}
    \begin{center}
      \ABNTEXchapterfont\bfseries\Large\imprimirtitulo
    \end{center}
    \vspace*{\fill}
    
    \hspace{.45\textwidth}
    \begin{minipage}{.5\textwidth}
        \imprimirpreambulo
    \end{minipage}%
    \vspace*{\fill}
   \end{center}
        
   Trabalho aprovado. \imprimirlocal, xx de julho de 2017:

   \assinatura{\textbf{Profª. Drª. \imprimirorientador} \\ Orientador} 
   \assinatura{\textbf{Prof} \\ Banca Examinadora}
   %\assinatura{\textbf{Professor} \\ Convidado 3}
   %\assinatura{\textbf{Professor} \\ Convidado 4}
      
   \begin{center}
    \vspace*{0.5cm}
    {\large\imprimirlocal}
    \par
    {\large\imprimirdata}
    \vspace*{1cm}
  \end{center}
  
\end{folhadeaprovacao}
% ---



% ---
% Agradecimentos
% ---
\begin{agradecimentos}


\end{agradecimentos}
% ---

% ---
% Epígrafe
% ---
\begin{epigrafe}
    \vspace*{\fill}
	\begin{flushright}
		\textit{``Grito aflito na rua do sossego'' \\
		(Alceu Valença)}
	\end{flushright}
\end{epigrafe}
% ---

% ---
% RESUMOS
% ---

% resumo em português
\setlength{\absparsep}{18pt} % ajusta o espaçamento dos parágrafos do resumo
\begin{resumo}
 

 \textbf{Palavras-chave}: 
\end{resumo}


% ---
% inserir lista de ilustrações
% ---
\pdfbookmark[0]{\listfigurename}{lof}
\listoffigures*
\cleardoublepage
% ---

% ---
% inserir lista de tabelas
% ---
\pdfbookmark[0]{\listtablename}{lot}
\listoftables*
\cleardoublepage
% ---

% ---
% inserir lista de abreviaturas e siglas
% ---
\begin{siglas}
  \item[ABNT] Associação Brasileira de Normas Técnicas
  \item[abnTeX] ABsurdas Normas para TeX
\end{siglas}
% ---

% ---
% inserir lista de símbolos
% ---
\begin{simbolos}
  \item[$ \Gamma $] Letra grega Gama
  \item[$ \Lambda $] Lambda
  \item[$ \zeta $] Letra grega minúscula zeta
  \item[$ \in $] Pertence
\end{simbolos}
% ---

% ---
% inserir o sumario
% ---
\pdfbookmark[0]{\contentsname}{toc}
\tableofcontents*
\cleardoublepage
% ---



% ----------------------------------------------------------
% ELEMENTOS TEXTUAIS
% ----------------------------------------------------------
\textual

% ---
% Introdução
% ---
\chapter{Introdução}
% ---

% ---
% Programa de Prestação Continuada e sua judicialização
% ---
\chapter{Programa de Prestação Continuada e sua judicialização}
O Benefício de Prestação Continuada é a garantia de um salário mínimo mensal, transferido de modo incondicional e idenpedente de qualquer contribuição prévia para o sistema de seguridade social. O benefício é destinado ao idoso com mais de 65 anos de idade e pessoa com deficiência que comprovem não possuir meios de prover a própria manutenção ou de tê-la provida pelos membros familiares residentes no mesmo domicílio. O benefício faz parte da política de assistência social, coordenado pelo Ministério do Desenvolvimento Social e Combate à Fome -- MDS, operacionalizado pelo Instituto Nacional do Seguro Social -- INSS e tem-se assegurado na constituição. Não obstante, sua regulamentação só ocorreu em 1993 na Lei Orgânica da Assistência Social (Loas), sendo implantada em 1996 após o Decreto n. 1744/1995 \cite{bpc_stf}. As regras de legebilidade foram redefinidas diversas vezes, a apresentada aqui será a mais atual, estabelecida peloXXXXXXXXXXXXXXXXXX


	\section{Regras de elegibilidade }
		\subsection{Carência de meios para a manutenção}
		A LOAS define quatro critérios para a definição dos elegíveis ao benefício: insuficiência de meios de provimento, conceito familiar, definição de idoso e  . A incapacidade de provimento mínimo é definida como a pessoa que tenha renda \textit{per capita} inferior a 1/4 do salário mínimo vigente. Ademais, nos casos de concessão do benefício para pessoas com deficiências, essas não poderão exercer atividade remunerada, excedo na condição de aprendiz por um prazo máximo de dois anos. Segundo \citeonline{adriana2016}, esse critério é largamente utilizado nos programas governamentais, de modo a facilitar a operacionalização dos programas e evitar o tratamento não isonômico. Alem disso, quando instituído o salário mínimo na constituição de 1998, sua definição presumia que o valor era capaz de atender às necessidades básicas de uma família mononuclear, ou seja: pessoa, conjugue e dois filhos, corroborando o corte de 1/4 do salário mínimo instituído.
		

		O conceito de família definido pelo programa é determinante para o cálculo da renda mínima de elegibilidade. A LOAS explicita a inclusão nesse cálculo as rendas do requerente ao benefício, seu cônjuge, filhos e enteados solteiros, irmãos solteiros, os pais e, na ausência de um deles, a madrasta ou o padrasto e menores tutelados, desde que esses integrantes familiares vivam sobre o mesmo teto. Como explicitado, o conceito familiar no BPC não é definido estritamente segundo a existência de uma unidade de consumo. \citeonline{medeiros2009mudancca} afirma que a definição atual de família do programa traz distorssoes, podendo superestimar a renda de algumas famílias pobres ou subestimar a capacidade de prover o sustento de famílias que tenham filhos e irmãos casados ou demais parentes mais ricos. Como essa definição impacta diretamente o cálculo da renda \textit{per capita}, estudos já verificaram o efeito de uma mudança nesse conceito. A mudança que considere a unidade domiciliar de consumo como família, traria a exclusão e introdução de beneficiários, tendo em média efeito nulo líquido. No entanto, traria uma maior focalização do programa \cite{medeiros2009mudancca,fambpcfreitas}.
		
		O conceito de idoso segue atualmente o descrito no Estatuto do Idoso de 2003, que considera idoso pessoas com 65 anos ou mais. A definição de deficiência segue a Convenção sobre Direitos das Pessoas com Deficiência das Nações Unidas, que estabelece que o a deficiência está em constante transformação 
		
		
		
		
		
\section{Tendência e entendimento atual das judicializações}
% ---

% ---
% Pobreza e sua dimensões
% ---
\chapter{Pobreza e sua dimensões}
% ---

% ---
% Métodos e Procedimentos
% ---
\chapter{Métodos e procedimentos}
	\section{Dados}
	A Pesquisa Nacional de Saúde de 2013 (PNS), do Instituto Brasileiro de Geografia e Estatística (IBGE), foi escolhida para prover os dados necessários para estimar o número de elegíveis idosos e pessoas com deficiência do BPC e calcular suas rendas familiares \textit{per capita} e índices de pobreza multidimensional. A escolha é pautada na abrangência nacional e disponibilidade de informações necessárias aos objetivos descritos acima, tais como: relação de parentesco entre membros do domicílio, recebimento de aposentadoria, presença de deficiência por tipos, grau de limitação das atividades habituais causadas pela deficiência, rendas do indivíduo, informações sobre saúde e domicílio, etc.   
	O tamanho da amostra é de aproximadamente XX mi de indivíduos, representando uma população de XX milhões. A subpopulação de interesse são os indivíduos que são elegíveis ao recebimento do BPC, excluindo-se a regra de renda, e os indivíduos que compõem o mesmo domicílio do suposto beneficiário, somando umas população de 5,8 milhões, sendo 3,8 milhões de elegíveis .
	
	\section{Identificação dos pré elegíveis e reconstrução da família BPC}
	
	A Pesquisa nacional de saúde define o conceito de família como sendo um arranjo familiar domiciliar, consistindo em um conjunto de parentes que vivem sob o mesmo teto, eventualmente sendo adicionadas pessoas que compartilhem recursos ou despesas dentro desse domicílio. As relações de parentesco na PNS são definidas entre cada componente da família e a pessoa responsável pela Unidade familiar. O responsável pelo domicílio é eleito pelo próprio morador entrevistado. Descoincidentemente a PNS, o BPC usa uma definição de família em que o próprio beneficiário é colocado como a pessoa de referência. Ademais, a Lei Orgânica de Assistência Social elenca os possíveis parentes que podem fazer parte dessa família do beneficiário.
	
	Pode-se assumir que o conceito de família no BPC está contido na definição da PNS. No entanto, esses conceitos apresentam diversas complicações quanto a  possibilidade de comparação fidedigna. Primeiro, os conceitos de pessoa de referência não são os mesmos, de modo que para se obter o grupo familiar do BPC a partir da família PNS, deve-se supor que o beneficiário é a pessoa de referência e reclassificar os demais. Por consequência, apesar da PNS apresentar um conceito mais amplo, não há nenhuma forma de representação de relações entre os componentes do domicílio, exceto com o responsável. Ou seja, embora seja possível saber que há famílias conviventes, não é trivial reconstruir com precisão os demais laços entre os indivíduos do domicílio. Outra dificuldade se refere a  existência das categorias de outros parentes e não parentes, que inviabiliza qualquer suposição das relações desses com os demais. 
	
	A identificação dos beneficiários do BPC foi feita em duas etapas: a primeira foi a identificação dos pré elegíveis, em que há uma aplicação de filtros que reconstruíssem as regras de elegibilidade, exceto renda \textit{per capita}, dada a necessidade de captar beneficiários que estão acima do corte de elegibilidade; depois o cálculo da renda per capita familiar.
	
	A primeira regra a ser observada é o público alvo: idosos e pessoas com deficiência. Para captar os idosos, foi criada uma \textit{dummy} indicando se o indivíduo tem 65 anos ou mais e não era beneficiário no âmbito da seguridade social ( como, aposentadoria e pensão). No caso das pessoas com deficiência, além de não poderem ser beneficiários de aposentadorias ou pensões, o BPC explicita critérios para essa definição em duas etapas: uma avaliação médica e uma social, que investiga restrições provenientes da interação entre deficiência e o meio em que vive. Como esse critério é genérico e abstrato, utilizou-se de dois blocos de perguntas disponíveis na PNS: o primeiro identifica se o indivíduo tem alguma deficiência intelectual, física, auditiva e visual. O segundo refere-se ao grau de limitação das atividades habituais geradas por essas deficiências. Essas limitações são  classificadas como: não limita, um pouco, moderadamente, intensamente e muito intensamente/ não consegue. Para esse estudo foram considerados dois níveis, sendo um em que considera-se deficientes elegíveis pessoas com grau de limitação maior ou igual à moderado e outro maior ou igual à intensamente. A tabela \ref*{tab_resumo_regras} apresenta as principais regras de elegibilidade do programa.
	
	\begin{table}[h]
		\footnotesize
		\centering
		\caption{Regras de elegibilidade para o BPC}
		\label{tab_resumo_regras}
			\resizebox{\textwidth}{!}{
		\begin{tabular}{|m{7cm}|m{7cm}|}
			\hline
			\multicolumn{1}{|c|}{\textbf{Idoso}}                                                                                                                                                                    & \multicolumn{1}{c|}{\textbf{Pessoa com deficiência}}                                                                                                                                                    \\ \hline
			Mínimo de 65 anos                                                                                                                                                                                       & Condição incapacitante para a vida independente e para o trabalho atestada pela perícia médica e social do INSS                                                                                       \\ \hline
			Renda per capita familiar de até 1/4 de salário mínimo                                                                                                                                                  & Renda per capita familiar de até 1/4 de salário mínimo                                                                                                                                                  \\ \hline
			Não acumular com aposentadorias e pensão ou de outro regime, exceto com benefícios da assistência médica, pensões especiais de natureza indenizatória e remuneração advinda de contrato de aprendizagem & Não acumular com aposentadorias e pensão ou de outro regime, exceto com benefícios da assistência médica, pensões especiais de natureza indenizatória e remuneração advinda de contrato de aprendizagem \\ \hline
		\end{tabular}
	}
	\end{table}
	
	
	Após identificadas as pessoas pré elegíveis ao benefício, necessita-se indicar quem entraria em sua composição familiar para fins de cálculo de renda \textit{per capita}. O método aqui utilizado baseia-se na única informação de vínculo entre os indivíduos existente na PNS: a condição da pessoa no domicílio. Assim, foi criada de uma tabela verdade que refaz as relações tomando por hipótese que a pessoa identificada como pré elegível é a pessoa de referência. Depois, são refeitas as classificações dos demais indivíduos do domicílio usando as regras descritas na LOAS para cada posição hipotética do beneficiário. O método leva em consideração a posição original do beneficiário no domicílio, a posição dos demais indivíduos, estado civil e indicativo de quem é o pré eleito ao benefício. 
	
	A tabela XXXX contém 360 regras, podendo-se chegar a esse número tal que:
	\begin{equation}
	((13)Pos_{titular} \cdot  (14)Pos_{todos}  \cdot (2)Estado_{civil} )-4 = 360
	\end{equation}
	
	Onde $ Pos_{titular} $ são as posições passíveis de serem assumidas pelo beneficiário dentro do domicílio, $ Pos_{todos} $ são as posições possíveis dos demais indivíduos do domicílio e inclusive ele mesmo e $ Estado_{civil}  $ é a possibilidade de ser solteiro ou casado. Pode-se observar que para todas as possíveis condições no domicílio, exceto pessoa de referência e conjugê, podem haver dois ou mais indivíduos na mesma posição que a do beneficiário, uma onde ele é o próprio e as demais em que a pessoa tem a mesma condição que ele. Por isso há a redução de 4 ao fim da equação, referente às duas posições que não podem existir mais de um individuo na mesma condição, tanto para solteiro quanto para não solteiro.
	
	\subsection{Ambiguidades e tratamento}
	Dentro da PNS nao há nenhuma questão que investigue as relações familiares dentro de um domicílio entre os demais componentes, exceto o responsável pelo domicílio. Assim, algumas imputações foram feitas respeitando a restrição de estado civil. As condições descritas abaixo estão em relação ao responsável do domicílio.
		\begin{itemize}
			\item Enteado é filho ou enteado do cônjuge
			\item Filhos só do responsável ou de ambos são irmão dos Enteados
			\item Enteado é irmão de enteado
			\item Irmãos são filhos do Pai, mãe, padrasto ou madrasta
			\item Sogro(a) são casados entre si
		\end{itemize}
	No entanto, algumas das categorias apresentam pouco ou nenhum indicativo de relação de parentesco com os demais e por isso foram agrupadas como "outros", são elas: outro parente, agregados, conviventes, pensionistas, empregado doméstico e parente do empregado doméstico. Para esses casos foi considerada apenas que eles não fazem parte da família BPC do responsável pelo domicílio e nem esse faria partes daqueles. O restante das reclassificações cruzadas para essas pessoas foram marcadas como ambíguas. Outras marcações ambíguas também foram feitas quando não foi possível sequer fazer imputação. A tabela \ref*{tab_resumo_reclass} resume as regras de reclassificações e indica as ambiguidades. 
	
	
\begin{table}[h]
			\footnotesize
	\centering
	\caption{Membros da Família BPC segundo a relação do mesmo com o responsável pelo domicílio na PNS }
	\label{tab_resumo_reclass}
	\resizebox{\textwidth}{!}{%
		\begin{tabular}{@{}p{2cm}lllllllllll@{}}
			\toprule
			Condição no domicílio (PNS) & \multicolumn{11}{c}{Membros da Família BPC}                                                                                       \\ \midrule
			& Responsável & Cônjuge & Filhos/enteados & Genro/Nora & Pais & Sogro(a) & Neto(a) & Bisneto(a) & Irmão/Irmã & Avô ou avó & outros  \\
			Responsável                                 &             & sim     & sim*            & não        & sim  & não      & não     & não        & sim*       & não        & não     \\
			Cônjuge                                     & sim         &         & sim*            & não        & não  & sim      & não     & não        & não        & não        & ambiguo \\
			Filhos/enteados                             & sim         & sim     & sim*            & ambiguo    & não  & não      & ambiguo & não        & não        & não        & ambiguo \\
			Genro/Nora                                  & não         & não     & ambiguo         & não        & não  & não      & ambiguo & não        & não        & não        & ambiguo \\
			Pais                                        & sim*        & não     & não             & não        & sim  & não      & não     & não        & sim*       & não        & ambiguo \\
			Sogro(a)                                    & não         & não     & não             & não        & não  & sim      & não     & não        & não        & não        & ambiguo \\
			Neto(a)                                     & não         & não     & ambiguo         & ambiguo    & não  & não      & ambiguo & ambiguo    & não        & não        & ambiguo \\
			Bisneto(a)                                  & não         & não     & não             & não        & não  & não      & ambiguo & ambiguo    & não        & não        & ambiguo \\
			Irmão/Irmã                                  & sim*        & não     & não             & não        & sim  & não      & não     & não        & sim*       & não        & ambiguo \\
			Avô ou avó                                  & não         & não     & não             & não        & não  & não      & não     & não        & não        & ambiguo    & ambiguo \\ \bottomrule
		\end{tabular}%
	}
\end{table}


	
	O anexo \ref{anexo_reclass} apresenta todas as 360 regras de reclassificação.
	
	\section{Cálculo da renda \textit{per capita} familiar}
	A LOAS identifica algumas fontes de renda que não podem ser computadas, as principais são: renda de trabalho na posição de aprendiz, renda do BPC de um idoso no cômputo da renda de outro idoso da mesma família e rendimentos provenientes de Bolsa Família. Os rendimentos provenientes do Programa Bolsa Família foram identificados por valores típicos. Para isso, foi utilizado o método seguido por \cite{metodologiaOsorio2011}. Depois de obtida a renda bruta familiar, essa foi dividida pelo número de pessoas que fazem parte do grupo familiar do beneficiário, encontrado pelo método de reclassificação. A partir disso, foram criados dois grupos: o primeiro de até 1/4 de salário mínimo \textit{per capita}), o qual respeita as regras atuais de elegibilidade; e outro maior que 1/4 e menor que 1/2.
	
	\section{Comparativo de pobreza entre grupos}
% ---

% ---
% Resultados
% ---
\chapter{Resultados}
% ---
\section{Análise Descritiva}
Em 2013 no Brasil, segundo a Pesquisa Nacional de Saúde (PNS), havia 17,9 milhões de pessoas com 65 anos ou mais (8,9\%) e 14,7 milhões de pessoas que diziam ter alguma deficiência intelectual, visual, física ou auditiva (7,3\%) com qualquer grau de limitação advinda dessas deficiências. A distribuição percentual destas populações por condição no domicílio, é apresentada na tabela \ref*{tab_prop_byc004}.

A categoria de pessoa responsável pelo domicílio agregam a maioria das pessoas, no entanto há uma diferença entre idosos e deficientes. Os idosos estão concentrados majoritariamente em três categorias: pessoa responsável pelo domicílio, cônjuge ou companheiro e pai, mãe, padrasto ou madrasta; 61\%, 22\% e 10\%, respectivamente. As pessoas com deficiência estão em mais categorias, concentrando-se, como mais de 90\%, em 4 categorias: responsável pelo domicílio, cônjuge, filhos e pai, pai, mãe, padrasto ou madrasta; 46\%, 22\%, 10,5\% e 4\%, respectivamente.

\begin{table}[h]
	\footnotesize
	\centering
	\caption{Brasil -- Distribuição percentual da população de 65 anos ou mais e pessoas com alguma deficiência, segundo a condição no domicílio, 2013}
	\label{tab_prop_byc004}
	\begin{tabular}{@{}lm{4cm}m{3cm}@{}}
		\toprule
		\textbf{Condição no domicílio}                         & \textbf{Pessoa com Deficiência} & \textbf{Idoso}  \\ \midrule
		Pessoa responsável pelo domicílio                      & 46,49                           & 61,49           \\
		Cônjuge ou companheiro(a) de sexo diferente            & 21,98                           & 22,03           \\
		Cônjuge ou companheiro(a) do mesmo sexo                & 0,02                            & 0               \\
		Filho(a) do responsável e do cônjuge                   & 10,51                           & 0,04            \\
		Filho(a) somente do responsável                        & 8,17                            & 0,17            \\
		Enteado(a)                                             & 0,93                            & 0               \\
		Genro ou nora                                          & 0,21                            & 0,07            \\
		Pai, mãe, padrasto ou madrasta                         & 4,34                            & 9,86            \\
		Sogro(a)                                               & 0,94                            & 2,41            \\
		Neto(a)                                                & 1,73                            & 0               \\
		Bisneto(a)                                             & 0,01                            & 0               \\
		Irmão ou irmã                                          & 2,37                            & 1,67            \\
		Avô ou avó                                             & 0,18                            & 0,64            \\
		Outro parente                                          & 1,55                            & 1,04            \\
		Agregado(a) – Não parente que não compartilha despesas & 0,19                            & 0,24            \\
		Convivente – Não parente que compartilha despesas      & 0,29                            & 0,26            \\
		Pensionista                                            & 0,05                            & 0,04            \\
		Empregado(a) doméstico(a)                              & 0,03                            & 0,02            \\ \midrule
		\textbf{Total}                                         & \textbf{100,00}                 & \textbf{100,00} \\ \bottomrule
	\end{tabular}
\end{table}

Quando selecionada as pessoas preeleitas ao recebimento do benefício, ou seja, aqueles que respeitaram todas as regras de elegibilidade do BPC, exceto de renda, foi encontrado um total de 4,2 milhões de pessoas. Desse total, 2,4 milhões foram classificados como espécie BPC idoso e 1,8 milhões de BPC deficiente, 58\% e 41\%, respectivamente. A distribuição percentual destas populações, por condição no domicílio e espécie do benefício, é apresentada na tabela \ref*{tab_prop_byc004_preeleito}


Quase 90\% do total de preeleitos estão na condições de responsável pelo domicílio, cônjuge, filhos ou pai, mãe, padrasto ou madrasta. Esse dado é relevante: essas posições são as que apresentam maior nível de acurácia na determinação do grupamento familiar BPC do beneficiário. A posição de menor nível de garantia de identificação do grupamento familiar estão em "outros" e representam  menos de 3\% do total. Subdividindo-se por espécie do benefício (se destinada à pessoa com deficiência ou ao idoso), a categoria de idoso apresenta ainda melhor situação para a reclassificação: mais de 90\% estão em três categorias e menos de 2\% estao em ``outro''. A despeito de as pessoas com deficiência estarem em mais categorias, ainda predominam-se, com 85\%, nas categorias de melhor nível de classificação: responsável pelo domicílio, cônjuge e filhos.
 
\begin{table}[h]
	\footnotesize
	\centering
	\caption{Brasil -- Distribuição percentual da população preeleita segundo a condição no domicílio e espécie do benefício, 2013}
	\label{tab_prop_byc004_preeleito}
	\begin{tabular}{@{}llp{3cm}p{3cm}@{}}
		\toprule
		\textbf{Condição no domicílio}       & \textbf{Pessoa com Deficiência} & \textbf{Idoso} & \textbf{Total} \\ \midrule
		Pessoa responsável pelo domicílio    & 21,85                           & 46,1           & 35,95          \\
		Cônjuge                              & 14,8                            & 38,27          & 28,45          \\
		Filho(a) do responsável e do cônjuge & 25,59                           & 0,01           & 10,71          \\
		Filho(a) somente do responsável      & 20,41                           & 0,23           & 8,67           \\
		Enteado(a)                           & 2,95                            & 0              & 1,23           \\
		Genro ou nora                        & 0,03                            & 0,11           & 0,07           \\
		Pai, mãe, padrasto ou madrasta       & 0,64                            & 9,06           & 5,54           \\
		Sogro(a)                             & 0,16                            & 2,07           & 1,27           \\
		Neto(a)                              & 5,52                            & 0              & 2,31           \\
		Bisneto(a)                           & 0,01                            & 0              & 0              \\
		Irmão ou irmã                        & 4,09                            & 1,73           & 2,71           \\
		Avô ou avó                           & 0                               & 0,76           & 0,44           \\
		Outros                               & 3,96                            & 1,66           & 2,62           \\ \midrule
		\textbf{Total}                       & \textbf{100}                    & \textbf{100}   & \textbf{100}   \\ \bottomrule
	\end{tabular}
\end{table}

A soma total de indivíduos passiveis de serem reclassificados como grupo familiar do BPC é de 6,4 milhões. A taxa de reclassificação total foi de 98,5\%. Quando subdividido por espécie do benefício, as famílias com o beneficiário idoso atingiu 99,4\% de reclassificação e as pessoas com alguma deficiência com restrição moderada tiveram 96,2\%. A menor taxa de sucesso nas reclassificações para beneficio ao deficiente era esperado, dada a mair ocorrencia de beneficiários em posições de menor acurácia na reconstrução de seus laços familiares e por consequência, maior ocorrência de ambiguidades. A tabela \ref*{tab_per_reclas} apresenta o resultado. 

\begin{table}[h]
	\footnotesize
	\centering
	\caption{Brasil -- Percentual de reclassificações por tipo, 2013}
	\label{tab_per_reclas}
	\begin{tabular}{@{}p{4.5cm}p{4cm}p{3cm}p{3cm}@{}}
		\toprule
		\textbf{Situação da reclassificação}     & \textbf{Pessoa com Deficiência} & \textbf{Idoso} & \textbf{Total} \\ \midrule
		Reclassificado                           & 96,22                  & 99,44 & 98,55 \\                 
		\multicolumn{1}{r}{Compoe a família}     & 91,11                  & 95,45 & 94,17 \\               
		\multicolumn{1}{r}{Não compoe a família} & 5,11                   & 4,00  & 4,38  \\               
		Ambíguo                                  & 3,78                   & 0,56  & 1,45  \\ \midrule
		Total                                    & 100                    & 100   & 100   \\ \bottomrule
	\end{tabular}
\end{table}

Quando excluídas as pessoas que não entram na família BPC e os indivíduos com reclassificação ambígua,a média de pessoas por família é de 1,7 pessoas, similar ao encontrado na estatística oficial do programa CITAR DOCUMENTO DE COMPROVAçãoo--------. O próprio beneficiário representa 65\% do total seguido de cônjuge com 18\%. Para o benefício destinado à pessoas com deficiência, essa estatística se modifica, sendo majoritariamente formada pelo próprio beneficiário e seus filhos. A distribuição da população reclassificada de acordo com a classificação do grupo familiar BPC está na tabela \ref*{tab_reclass_cond}.

\begin{table}[h]
	\footnotesize
	\centering
	\caption{Brasil -- Distribuição percentual da população reclassificada por condição da família BPC, 2013}
	\label{tab_reclass_cond}
	\begin{tabular}{@{}p{4.5cm}p{4cm}p{3cm}p{3cm}@{}}
		\toprule
		\textbf{Classificação família BPC} & \textbf{Pessoa com Deficiência} & \textbf{Idoso} & \textbf{Total} \\ \midrule
		Beneficiário                       & 67,07                           & 63,72          & 65,08          \\
		Cônjuge/companheiro(a)             & 4,94                            & 28,18          & 18,81          \\
		Filhos                             & 14,74                           & 2,45           & 2,10           \\
		Pai/mãe/madrasta/padrasto          & 2,75                            & 0,55           & 6,31           \\
		Ambíguo                            & 3,78                            & 0,56           & 1,45           \\
		Irmãos                             & 1,59                            & 0,55           & 1,86           \\
		Enteados                           & 0,03                            & 0,00           & 0,01           \\
		Não entra                          & 5,11                            & 4,00           & 4,38           \\ \midrule
		Total                              & 100,00                          & 100,00         & 100,00         \\ \bottomrule
	\end{tabular}
\end{table}


% ---
% Conclusão
% ---
\chapter{Conclusão}
% ---



% ----------------------------------------------------------
% ELEMENTOS PÓS-TEXTUAIS
% ----------------------------------------------------------
\postextual
% ----------------------------------------------------------

% ----------------------------------------------------------
% Referências bibliográficas
% ----------------------------------------------------------
\bibliography{bibliografia}

% ---
% Inicia os anexos
% ---
\begin{anexosenv}
	
	% Imprime uma página indicando o início dos anexos
	\partanexos
	
	% ---
	\chapter{Tabela verdade de recriação do grupamento familiar do BPC}
	\label{anexo_reclass}
	\footnotesize
	\begin{longtable}{@{}lcclc@{}}
			\toprule
			Condição do pré eleito ao BPC        & Pré elegivel & Solteiro & Condição no domicílio                & Vinculo familiar BPC \\ \midrule
			\endfirsthead
			\multicolumn{5}{c}%
			{\tablename\ \thetable\ -- \textit{Continuação da tabela}} \\
			\toprule
			Condição do pré eleito ao BPC        & Pré elegivel & Solteiro & Condição no domicílio                & Vinculo familiar BPC \\ \midrule
			\endhead
			\hline \multicolumn{5}{r}{\textit{Continua na próxima página}} \\
			\endfoot
			\hline
			\endlastfoot
				Pessoa responsável pelo domicílio & 1 & 1 & Pessoa responsável pelo domicílio & O PROPRIO \\
				Pessoa responsável pelo domicílio & 0 & 1 & Conjuge & ERRO \\
				Pessoa responsável pelo domicílio & 0 & 1 & Filho(a) do responsável e do cônjuge & FILHO \\
				Pessoa responsável pelo domicílio & 0 & 1 & Filho(a) somente do responsável & FILHO \\
				Pessoa responsável pelo domicílio & 0 & 1 & Enteado(a) & ENTEADO \\
				Pessoa responsável pelo domicílio & 0 & 1 & Genro ou nora & ERRO \\
				Pessoa responsável pelo domicílio & 0 & 1 & Pai, mãe, padrasto ou madrasta & PAI/MÃE \\
				Pessoa responsável pelo domicílio & 0 & 1 & Sogro(a) & NÃO ENTRA \\
				Pessoa responsável pelo domicílio & 0 & 1 & Neto(a) & NÃO ENTRA \\
				Pessoa responsável pelo domicílio & 0 & 1 & Bisneto(a) & NÃO ENTRA \\
				Pessoa responsável pelo domicílio & 0 & 1 & Irmão ou irmã & IRMAO/IRMA \\
				Pessoa responsável pelo domicílio & 0 & 1 & Avô ou avó & NÃO ENTRA \\
				Pessoa responsável pelo domicílio & 0 & 1 & Outros & NÃO ENTRA \\
				Conjuge & 0 & 1 & Pessoa responsável pelo domicílio & ERRO \\
				Conjuge & 1 & 1 & Conjuge & ERRO \\
				Conjuge & 0 & 1 & Filho(a) do responsável e do cônjuge & FILHO \\
				Conjuge & 0 & 1 & Filho(a) somente do responsável & ENTEADO \\
				Conjuge & 0 & 1 & Enteado(a) & AMBIGUO \\
				Conjuge & 0 & 1 & Genro ou nora & ERRO \\
				Conjuge & 0 & 1 & Pai, mãe, padrasto ou madrasta & NÃO ENTRA \\
				Conjuge & 0 & 1 & Sogro(a) & PAI/MÃE \\
				Conjuge & 0 & 1 & Neto(a) & NÃO ENTRA \\
				Conjuge & 0 & 1 & Bisneto(a) & NÃO ENTRA \\
				Conjuge & 0 & 1 & Irmão ou irmã & NÃO ENTRA \\
				Conjuge & 0 & 1 & Avô ou avó & NÃO ENTRA \\
				Conjuge & 0 & 1 & Outros & AMBIGUO \\
				Filho(a) do responsável e do cônjuge & 1 & 1 & Filho(a) do responsável e do cônjuge & O PROPRIO \\
				Filho(a) do responsável e do cônjuge & 0 & 1 & Pessoa responsável pelo domicílio & PAI/MÃE \\
				Filho(a) do responsável e do cônjuge & 0 & 1 & Conjuge & PAI/MÃE \\
				Filho(a) do responsável e do cônjuge & 0 & 1 & Filho(a) do responsável e do cônjuge & IRMAO/IRMA \\
				Filho(a) do responsável e do cônjuge & 0 & 1 & Filho(a) somente do responsável & IRMAO/IRMA \\
				Filho(a) do responsável e do cônjuge & 0 & 1 & Enteado(a) & IRMAO/IRMA \\
				Filho(a) do responsável e do cônjuge & 0 & 1 & Genro ou nora & ERRO \\
				Filho(a) do responsável e do cônjuge & 0 & 1 & Pai, mãe, padrasto ou madrasta & NÃO ENTRA \\
				Filho(a) do responsável e do cônjuge & 0 & 1 & Sogro(a) & NÃO ENTRA \\
				Filho(a) do responsável e do cônjuge & 0 & 1 & Neto(a) & AMBIGUO \\
				Filho(a) do responsável e do cônjuge & 0 & 1 & Bisneto(a) & NÃO ENTRA \\
				Filho(a) do responsável e do cônjuge & 0 & 1 & Irmão ou irmã & NÃO ENTRA \\
				Filho(a) do responsável e do cônjuge & 0 & 1 & Avô ou avó & NÃO ENTRA \\
				Filho(a) do responsável e do cônjuge & 0 & 1 & Outros & AMBIGUO \\
				Filho(a) somente do responsável & 1 & 1 & Filho(a) somente do responsável & O PROPRIO \\
				Filho(a) somente do responsável & 0 & 1 & Pessoa responsável pelo domicílio & PAI/MÃE \\
				Filho(a) somente do responsável & 0 & 1 & Conjuge & PAI/MÃE \\
				Filho(a) somente do responsável & 0 & 1 & Filho(a) do responsável e do cônjuge & IRMAO/IRMA \\
				Filho(a) somente do responsável & 0 & 1 & Filho(a) somente do responsável & IRMAO/IRMA \\
				Filho(a) somente do responsável & 0 & 1 & Enteado(a) & IRMAO/IRMA \\
				Filho(a) somente do responsável & 0 & 1 & Genro ou nora & ERRO \\
				Filho(a) somente do responsável & 0 & 1 & Pai, mãe, padrasto ou madrasta & NÃO ENTRA \\
				Filho(a) somente do responsável & 0 & 1 & Sogro(a) & NÃO ENTRA \\
				Filho(a) somente do responsável & 0 & 1 & Neto(a) & AMBIGUO \\
				Filho(a) somente do responsável & 0 & 1 & Bisneto(a) & NÃO ENTRA \\
				Filho(a) somente do responsável & 0 & 1 & Irmão ou irmã & NÃO ENTRA \\
				Filho(a) somente do responsável & 0 & 1 & Avô ou avó & NÃO ENTRA \\
				Filho(a) somente do responsável & 0 & 1 & Outros & AMBIGUO \\
				Enteado(a) & 1 & 1 & Enteado(a) & O PROPRIO \\
				Enteado(a) & 0 & 1 & Pessoa responsável pelo domicílio & PAI/MÃE \\
				Enteado(a) & 0 & 1 & Conjuge & ERRO \\
				Enteado(a) & 0 & 1 & Filho(a) do responsável e do cônjuge & IRMAO/IRMA \\
				Enteado(a) & 0 & 1 & Filho(a) somente do responsável & IRMAO/IRMA \\
				Enteado(a) & 0 & 1 & Enteado(a) & AMBIGUO \\
				Enteado(a) & 0 & 1 & Genro ou nora & ERRO \\
				Enteado(a) & 0 & 1 & Pai, mãe, padrasto ou madrasta & NÃO ENTRA \\
				Enteado(a) & 0 & 1 & Sogro(a) & NÃO ENTRA \\
				Enteado(a) & 0 & 1 & Neto(a) & AMBIGUO \\
				Enteado(a) & 0 & 1 & Bisneto(a) & NÃO ENTRA \\
				Enteado(a) & 0 & 1 & Irmão ou irmã & NÃO ENTRA \\
				Enteado(a) & 0 & 1 & Avô ou avó & NÃO ENTRA \\
				Enteado(a) & 0 & 1 & Outros & AMBIGUO \\
				Genro ou nora & 1 & 1 & Genro ou nora & ERRO \\
				Genro ou nora & 0 & 1 & Pessoa responsável pelo domicílio & NÃO ENTRA \\
				Genro ou nora & 0 & 1 & Conjuge & ERRO \\
				Genro ou nora & 0 & 1 & Filho(a) do responsável e do cônjuge & AMBIGUO \\
				Genro ou nora & 0 & 1 & Filho(a) somente do responsável & AMBIGUO \\
				Genro ou nora & 0 & 1 & Enteado(a) & AMBIGUO \\
				Genro ou nora & 0 & 1 & Genro ou nora & ERRO \\
				Genro ou nora & 0 & 1 & Pai, mãe, padrasto ou madrasta & NÃO ENTRA \\
				Genro ou nora & 0 & 1 & Sogro(a) & NÃO ENTRA \\
				Genro ou nora & 0 & 1 & Neto(a) & AMBIGUO \\
				Genro ou nora & 0 & 1 & Bisneto(a) & NÃO ENTRA \\
				Genro ou nora & 0 & 1 & Irmão ou irmã & NÃO ENTRA \\
				Genro ou nora & 0 & 1 & Avô ou avó & NÃO ENTRA \\
				Genro ou nora & 0 & 1 & Outros & AMBIGUO \\
				Pai, mãe, padrasto ou madrasta & 1 & 1 & Pai, mãe, padrasto ou madrasta & O PROPRIO \\
				Pai, mãe, padrasto ou madrasta & 0 & 1 & Pessoa responsável pelo domicílio & FILHO \\
				Pai, mãe, padrasto ou madrasta & 0 & 1 & Conjuge & ERRO \\
				Pai, mãe, padrasto ou madrasta & 0 & 1 & Filho(a) do responsável e do cônjuge & NÃO ENTRA \\
				Pai, mãe, padrasto ou madrasta & 0 & 1 & Filho(a) somente do responsável & NÃO ENTRA \\
				Pai, mãe, padrasto ou madrasta & 0 & 1 & Enteado(a) & NÃO ENTRA \\
				Pai, mãe, padrasto ou madrasta & 0 & 1 & Genro ou nora & NÃO ENTRA \\
				Pai, mãe, padrasto ou madrasta & 0 & 1 & Pai, mãe, padrasto ou madrasta & NÃO ENTRA \\
				Pai, mãe, padrasto ou madrasta & 0 & 1 & Sogro(a) & NÃO ENTRA \\
				Pai, mãe, padrasto ou madrasta & 0 & 1 & Neto(a) & NÃO ENTRA \\
				Pai, mãe, padrasto ou madrasta & 0 & 1 & Bisneto(a) & NÃO ENTRA \\
				Pai, mãe, padrasto ou madrasta & 0 & 1 & Irmão ou irmã & NÃO ENTRA \\
				Pai, mãe, padrasto ou madrasta & 0 & 1 & Avô ou avó & PAI/MÃE \\
				Pai, mãe, padrasto ou madrasta & 0 & 1 & Outros & AMBIGUO \\
				Sogro(a) & 1 & 1 & Sogro(a) & O PROPRIO \\
				Sogro(a) & 0 & 1 & Pessoa responsável pelo domicílio & NÃO ENTRA \\
				Sogro(a) & 0 & 1 & Conjuge & ERRO \\
				Sogro(a) & 0 & 1 & Filho(a) do responsável e do cônjuge & NÃO ENTRA \\
				Sogro(a) & 0 & 1 & Filho(a) somente do responsável & NÃO ENTRA \\
				Sogro(a) & 0 & 1 & Enteado(a) & NÃO ENTRA \\
				Sogro(a) & 0 & 1 & Genro ou nora & NÃO ENTRA \\
				Sogro(a) & 0 & 1 & Pai, mãe, padrasto ou madrasta & NÃO ENTRA \\
				Sogro(a) & 0 & 1 & Sogro(a) & NÃO ENTRA \\
				Sogro(a) & 0 & 1 & Neto(a) & NÃO ENTRA \\
				Sogro(a) & 0 & 1 & Bisneto(a) & NÃO ENTRA \\
				Sogro(a) & 0 & 1 & Irmão ou irmã & NÃO ENTRA \\
				Sogro(a) & 0 & 1 & Avô ou avó & NÃO ENTRA \\
				Sogro(a) & 0 & 1 & Outros & AMBIGUO \\
				Neto(a) & 1 & 1 & Neto(a) & O PROPRIO \\
				Neto(a) & 0 & 1 & Pessoa responsável pelo domicílio & NÃO ENTRA \\
				Neto(a) & 0 & 1 & Conjuge & ERRO \\
				Neto(a) & 0 & 1 & Filho(a) do responsável e do cônjuge & AMBIGUO \\
				Neto(a) & 0 & 1 & Filho(a) somente do responsável & AMBIGUO \\
				Neto(a) & 0 & 1 & Enteado(a) & AMBIGUO \\
				Neto(a) & 0 & 1 & Genro ou nora & ERRO \\
				Neto(a) & 0 & 1 & Pai, mãe, padrasto ou madrasta & NÃO ENTRA \\
				Neto(a) & 0 & 1 & Sogro(a) & NÃO ENTRA \\
				Neto(a) & 0 & 1 & Neto(a) & AMBIGUO \\
				Neto(a) & 0 & 1 & Bisneto(a) & NÃO ENTRA \\
				Neto(a) & 0 & 1 & Irmão ou irmã & NÃO ENTRA \\
				Neto(a) & 0 & 1 & Avô ou avó & NÃO ENTRA \\
				Neto(a) & 0 & 1 & Outros & AMBIGUO \\
				Bisneto(a) & 1 & 1 & Bisneto(a) & O PROPRIO \\
				Bisneto(a) & 0 & 1 & Pessoa responsável pelo domicílio & NÃO ENTRA \\
				Bisneto(a) & 0 & 1 & Conjuge & ERRO \\
				Bisneto(a) & 0 & 1 & Filho(a) do responsável e do cônjuge & NÃO ENTRA \\
				Bisneto(a) & 0 & 1 & Filho(a) somente do responsável & NÃO ENTRA \\
				Bisneto(a) & 0 & 1 & Enteado(a) & NÃO ENTRA \\
				Bisneto(a) & 0 & 1 & Genro ou nora & ERRO \\
				Bisneto(a) & 0 & 1 & Pai, mãe, padrasto ou madrasta & NÃO ENTRA \\
				Bisneto(a) & 0 & 1 & Sogro(a) & NÃO ENTRA \\
				Bisneto(a) & 0 & 1 & Neto(a) & AMBIGUO \\
				Bisneto(a) & 0 & 1 & Bisneto(a) & AMBIGUO \\
				Bisneto(a) & 0 & 1 & Irmão ou irmã & NÃO ENTRA \\
				Bisneto(a) & 0 & 1 & Avô ou avó & NÃO ENTRA \\
				Bisneto(a) & 0 & 1 & Outros & AMBIGUO \\
				Irmão ou irmã & 1 & 1 & Irmão ou irmã & O PROPRIO \\
				Irmão ou irmã & 0 & 1 & Pessoa responsável pelo domicílio & IRMAO/IRMA \\
				Irmão ou irmã & 0 & 1 & Conjuge & ERRO \\
				Irmão ou irmã & 0 & 1 & Filho(a) do responsável e do cônjuge & NÃO ENTRA \\
				Irmão ou irmã & 0 & 1 & Filho(a) somente do responsável & NÃO ENTRA \\
				Irmão ou irmã & 0 & 1 & Enteado(a) & NÃO ENTRA \\
				Irmão ou irmã & 0 & 1 & Genro ou nora & ERRO \\
				Irmão ou irmã & 0 & 1 & Pai, mãe, padrasto ou madrasta & PAI/MÃE \\
				Irmão ou irmã & 0 & 1 & Sogro(a) & NÃO ENTRA \\
				Irmão ou irmã & 0 & 1 & Neto(a) & NÃO ENTRA \\
				Irmão ou irmã & 0 & 1 & Bisneto(a) & NÃO ENTRA \\
				Irmão ou irmã & 0 & 1 & Irmão ou irmã & IRMAO/IRMA \\
				Irmão ou irmã & 0 & 1 & Avô ou avó & NÃO ENTRA \\
				Irmão ou irmã & 0 & 1 & Outros & AMBIGUO \\
				Avô ou avó & 1 & 1 & Avô ou avó & O PROPRIO \\
				Avô ou avó & 0 & 1 & Pessoa responsável pelo domicílio & NÃO ENTRA \\
				Avô ou avó & 0 & 1 & Conjuge & ERRO \\
				Avô ou avó & 0 & 1 & Filho(a) do responsável e do cônjuge & NÃO ENTRA \\
				Avô ou avó & 0 & 1 & Filho(a) somente do responsável & NÃO ENTRA \\
				Avô ou avó & 0 & 1 & Enteado(a) & NÃO ENTRA \\
				Avô ou avó & 0 & 1 & Genro ou nora & ERRO \\
				Avô ou avó & 0 & 1 & Pai, mãe, padrasto ou madrasta & AMBIGUO \\
				Avô ou avó & 0 & 1 & Sogro(a) & NÃO ENTRA \\
				Avô ou avó & 0 & 1 & Neto(a) & NÃO ENTRA \\
				Avô ou avó & 0 & 1 & Bisneto(a) & NÃO ENTRA \\
				Avô ou avó & 0 & 1 & Irmão ou irmã & NÃO ENTRA \\
				Avô ou avó & 0 & 1 & Avô ou avó & NÃO ENTRA \\
				Avô ou avó & 0 & 1 & Outros & AMBIGUO \\
				Outros & 1 & 1 & Outros & O PROPRIO \\
				Outros & 0 & 1 & Pessoa responsável pelo domicílio & NÃO ENTRA \\
				Outros & 0 & 1 & Conjuge & ERRO \\
				Outros & 0 & 1 & Filho(a) do responsável e do cônjuge & AMBIGUO \\
				Outros & 0 & 1 & Filho(a) somente do responsável & AMBIGUO \\
				Outros & 0 & 1 & Enteado(a) & AMBIGUO \\
				Outros & 0 & 1 & Genro ou nora & ERRO \\
				Outros & 0 & 1 & Pai, mãe, padrasto ou madrasta & AMBIGUO \\
				Outros & 0 & 1 & Sogro(a) & AMBIGUO \\
				Outros & 0 & 1 & Neto(a) & AMBIGUO \\
				Outros & 0 & 1 & Bisneto(a) & AMBIGUO \\
				Outros & 0 & 1 & Irmão ou irmã & AMBIGUO \\
				Outros & 0 & 1 & Avô ou avó & AMBIGUO \\
				Outros & 0 & 1 & Outros & AMBIGUO \\
				Pessoa responsável pelo domicílio & 1 & 0 & Pessoa responsável pelo domicílio & O PROPRIO \\
				Pessoa responsável pelo domicílio & 0 & 0 & Conjuge & CONJUGE \\
				Pessoa responsável pelo domicílio & 0 & 0 & Filho(a) do responsável e do cônjuge & NÃO ENTRA \\
				Pessoa responsável pelo domicílio & 0 & 0 & Filho(a) somente do responsável & NÃO ENTRA \\
				Pessoa responsável pelo domicílio & 0 & 0 & Enteado(a) & NÃO ENTRA \\
				Pessoa responsável pelo domicílio & 0 & 0 & Genro ou nora & NÃO ENTRA \\
				Pessoa responsável pelo domicílio & 0 & 0 & Pai, mãe, padrasto ou madrasta & PAI/MÃE \\
				Pessoa responsável pelo domicílio & 0 & 0 & Sogro(a) & NÃO ENTRA \\
				Pessoa responsável pelo domicílio & 0 & 0 & Neto(a) & NÃO ENTRA \\
				Pessoa responsável pelo domicílio & 0 & 0 & Bisneto(a) & NÃO ENTRA \\
				Pessoa responsável pelo domicílio & 0 & 0 & Irmão ou irmã & NÃO ENTRA \\
				Pessoa responsável pelo domicílio & 0 & 0 & Avô ou avó & NÃO ENTRA \\
				Pessoa responsável pelo domicílio & 0 & 0 & Outros & NÃO ENTRA \\
				Conjuge & 0 & 0 & Pessoa responsável pelo domicílio & CONJUGE \\
				Conjuge & 1 & 0 & Conjuge & O PROPRIO \\
				Conjuge & 0 & 0 & Filho(a) do responsável e do cônjuge & NÃO ENTRA \\
				Conjuge & 0 & 0 & Filho(a) somente do responsável & NÃO ENTRA \\
				Conjuge & 0 & 0 & Enteado(a) & NÃO ENTRA \\
				Conjuge & 0 & 0 & Genro ou nora & NÃO ENTRA \\
				Conjuge & 0 & 0 & Pai, mãe, padrasto ou madrasta & NÃO ENTRA \\
				Conjuge & 0 & 0 & Sogro(a) & PAI/MÃE \\
				Conjuge & 0 & 0 & Neto(a) & NÃO ENTRA \\
				Conjuge & 0 & 0 & Bisneto(a) & NÃO ENTRA \\
				Conjuge & 0 & 0 & Irmão ou irmã & NÃO ENTRA \\
				Conjuge & 0 & 0 & Avô ou avó & NÃO ENTRA \\
				Conjuge & 0 & 0 & Outros & AMBIGUO \\
				Filho(a) do responsável e do cônjuge & 1 & 0 & Filho(a) do responsável e do cônjuge & O PROPRIO \\
				Filho(a) do responsável e do cônjuge & 0 & 0 & Pessoa responsável pelo domicílio & PAI/MÃE \\
				Filho(a) do responsável e do cônjuge & 0 & 0 & Conjuge & PAI/MÃE \\
				Filho(a) do responsável e do cônjuge & 0 & 0 & Filho(a) do responsável e do cônjuge & NÃO ENTRA \\
				Filho(a) do responsável e do cônjuge & 0 & 0 & Filho(a) somente do responsável & NÃO ENTRA \\
				Filho(a) do responsável e do cônjuge & 0 & 0 & Enteado(a) & NÃO ENTRA \\
				Filho(a) do responsável e do cônjuge & 0 & 0 & Genro ou nora & AMBIGUO \\
				Filho(a) do responsável e do cônjuge & 0 & 0 & Pai, mãe, padrasto ou madrasta & NÃO ENTRA \\
				Filho(a) do responsável e do cônjuge & 0 & 0 & Sogro(a) & NÃO ENTRA \\
				Filho(a) do responsável e do cônjuge & 0 & 0 & Neto(a) & NÃO ENTRA \\
				Filho(a) do responsável e do cônjuge & 0 & 0 & Bisneto(a) & NÃO ENTRA \\
				Filho(a) do responsável e do cônjuge & 0 & 0 & Irmão ou irmã & NÃO ENTRA \\
				Filho(a) do responsável e do cônjuge & 0 & 0 & Avô ou avó & NÃO ENTRA \\
				Filho(a) do responsável e do cônjuge & 0 & 0 & Outros & AMBIGUO \\
				Filho(a) somente do responsável & 1 & 0 & Filho(a) somente do responsável & O PROPRIO \\
				Filho(a) somente do responsável & 0 & 0 & Pessoa responsável pelo domicílio & PAI/MÃE \\
				Filho(a) somente do responsável & 0 & 0 & Conjuge & PAI/MÃE \\
				Filho(a) somente do responsável & 0 & 0 & Filho(a) do responsável e do cônjuge & NÃO ENTRA \\
				Filho(a) somente do responsável & 0 & 0 & Filho(a) somente do responsável & NÃO ENTRA \\
				Filho(a) somente do responsável & 0 & 0 & Enteado(a) & NÃO ENTRA \\
				Filho(a) somente do responsável & 0 & 0 & Genro ou nora & AMBIGUO \\
				Filho(a) somente do responsável & 0 & 0 & Pai, mãe, padrasto ou madrasta & NÃO ENTRA \\
				Filho(a) somente do responsável & 0 & 0 & Sogro(a) & NÃO ENTRA \\
				Filho(a) somente do responsável & 0 & 0 & Neto(a) & NÃO ENTRA \\
				Filho(a) somente do responsável & 0 & 0 & Bisneto(a) & NÃO ENTRA \\
				Filho(a) somente do responsável & 0 & 0 & Irmão ou irmã & NÃO ENTRA \\
				Filho(a) somente do responsável & 0 & 0 & Avô ou avó & NÃO ENTRA \\
				Filho(a) somente do responsável & 0 & 0 & Outros & AMBIGUO \\
				Enteado(a) & 1 & 0 & Enteado(a) & O PROPRIO \\
				Enteado(a) & 0 & 0 & Pessoa responsável pelo domicílio & PAI/MÃE \\
				Enteado(a) & 0 & 0 & Conjuge & AMBIGUO \\
				Enteado(a) & 0 & 0 & Filho(a) do responsável e do cônjuge & NÃO ENTRA \\
				Enteado(a) & 0 & 0 & Filho(a) somente do responsável & NÃO ENTRA \\
				Enteado(a) & 0 & 0 & Enteado(a) & NÃO ENTRA \\
				Enteado(a) & 0 & 0 & Genro ou nora & NÃO ENTRA \\
				Enteado(a) & 0 & 0 & Pai, mãe, padrasto ou madrasta & NÃO ENTRA \\
				Enteado(a) & 0 & 0 & Sogro(a) & NÃO ENTRA \\
				Enteado(a) & 0 & 0 & Neto(a) & NÃO ENTRA \\
				Enteado(a) & 0 & 0 & Bisneto(a) & NÃO ENTRA \\
				Enteado(a) & 0 & 0 & Irmão ou irmã & NÃO ENTRA \\
				Enteado(a) & 0 & 0 & Avô ou avó & NÃO ENTRA \\
				Enteado(a) & 0 & 0 & Outros & AMBIGUO \\
				Genro ou nora & 1 & 0 & Genro ou nora & O PROPRIO \\
				Genro ou nora & 0 & 0 & Pessoa responsável pelo domicílio & NÃO ENTRA \\
				Genro ou nora & 0 & 0 & Conjuge & NÃO ENTRA \\
				Genro ou nora & 0 & 0 & Filho(a) do responsável e do cônjuge & AMBIGUO \\
				Genro ou nora & 0 & 0 & Filho(a) somente do responsável & AMBIGUO \\
				Genro ou nora & 0 & 0 & Enteado(a) & AMBIGUO \\
				Genro ou nora & 0 & 0 & Genro ou nora & NÃO ENTRA \\
				Genro ou nora & 0 & 0 & Pai, mãe, padrasto ou madrasta & NÃO ENTRA \\
				Genro ou nora & 0 & 0 & Sogro(a) & NÃO ENTRA \\
				Genro ou nora & 0 & 0 & Neto(a) & AMBIGUO \\
				Genro ou nora & 0 & 0 & Bisneto(a) & NÃO ENTRA \\
				Genro ou nora & 0 & 0 & Irmão ou irmã & NÃO ENTRA \\
				Genro ou nora & 0 & 0 & Avô ou avó & NÃO ENTRA \\
				Genro ou nora & 0 & 0 & Outros & AMBIGUO \\
				Pai, mãe, padrasto ou madrasta & 1 & 0 & Pai, mãe, padrasto ou madrasta & O PROPRIO \\
				Pai, mãe, padrasto ou madrasta & 0 & 0 & Pessoa responsável pelo domicílio & NÃO ENTRA \\
				Pai, mãe, padrasto ou madrasta & 0 & 0 & Conjuge & NÃO ENTRA \\
				Pai, mãe, padrasto ou madrasta & 0 & 0 & Filho(a) do responsável e do cônjuge & NÃO ENTRA \\
				Pai, mãe, padrasto ou madrasta & 0 & 0 & Filho(a) somente do responsável & NÃO ENTRA \\
				Pai, mãe, padrasto ou madrasta & 0 & 0 & Enteado(a) & NÃO ENTRA \\
				Pai, mãe, padrasto ou madrasta & 0 & 0 & Genro ou nora & NÃO ENTRA \\
				Pai, mãe, padrasto ou madrasta & 0 & 0 & Pai, mãe, padrasto ou madrasta & CONJUGE \\
				Pai, mãe, padrasto ou madrasta & 0 & 0 & Sogro(a) & NÃO ENTRA \\
				Pai, mãe, padrasto ou madrasta & 0 & 0 & Neto(a) & NÃO ENTRA \\
				Pai, mãe, padrasto ou madrasta & 0 & 0 & Bisneto(a) & NÃO ENTRA \\
				Pai, mãe, padrasto ou madrasta & 0 & 0 & Irmão ou irmã & NÃO ENTRA \\
				Pai, mãe, padrasto ou madrasta & 0 & 0 & Avô ou avó & PAI/MÃE \\
				Pai, mãe, padrasto ou madrasta & 0 & 0 & Outros & AMBIGUO \\
				Sogro(a) & 1 & 0 & Sogro(a) & O PROPRIO \\
				Sogro(a) & 0 & 0 & Pessoa responsável pelo domicílio & NÃO ENTRA \\
				Sogro(a) & 0 & 0 & Conjuge & NÃO ENTRA \\
				Sogro(a) & 0 & 0 & Filho(a) do responsável e do cônjuge & NÃO ENTRA \\
				Sogro(a) & 0 & 0 & Filho(a) somente do responsável & NÃO ENTRA \\
				Sogro(a) & 0 & 0 & Enteado(a) & NÃO ENTRA \\
				Sogro(a) & 0 & 0 & Genro ou nora & NÃO ENTRA \\
				Sogro(a) & 0 & 0 & Pai, mãe, padrasto ou madrasta & NÃO ENTRA \\
				Sogro(a) & 0 & 0 & Sogro(a) & CONJUGE \\
				Sogro(a) & 0 & 0 & Neto(a) & NÃO ENTRA \\
				Sogro(a) & 0 & 0 & Bisneto(a) & NÃO ENTRA \\
				Sogro(a) & 0 & 0 & Irmão ou irmã & NÃO ENTRA \\
				Sogro(a) & 0 & 0 & Avô ou avó & NÃO ENTRA \\
				Sogro(a) & 0 & 0 & Outros & AMBIGUO \\
				Neto(a) & 1 & 0 & Neto(a) & O PROPRIO \\
				Neto(a) & 0 & 0 & Pessoa responsável pelo domicílio & NÃO ENTRA \\
				Neto(a) & 0 & 0 & Conjuge & NÃO ENTRA \\
				Neto(a) & 0 & 0 & Filho(a) do responsável e do cônjuge & AMBIGUO \\
				Neto(a) & 0 & 0 & Filho(a) somente do responsável & AMBIGUO \\
				Neto(a) & 0 & 0 & Enteado(a) & AMBIGUO \\
				Neto(a) & 0 & 0 & Genro ou nora & AMBIGUO \\
				Neto(a) & 0 & 0 & Pai, mãe, padrasto ou madrasta & NÃO ENTRA \\
				Neto(a) & 0 & 0 & Sogro(a) & NÃO ENTRA \\
				Neto(a) & 0 & 0 & Neto(a) & NÃO ENTRA \\
				Neto(a) & 0 & 0 & Bisneto(a) & NÃO ENTRA \\
				Neto(a) & 0 & 0 & Irmão ou irmã & NÃO ENTRA \\
				Neto(a) & 0 & 0 & Avô ou avó & NÃO ENTRA \\
				Neto(a) & 0 & 0 & Outros & AMBIGUO \\
				Bisneto(a) & 1 & 0 & Bisneto(a) & O PROPRIO \\
				Bisneto(a) & 0 & 0 & Pessoa responsável pelo domicílio & NÃO ENTRA \\
				Bisneto(a) & 0 & 0 & Conjuge & NÃO ENTRA \\
				Bisneto(a) & 0 & 0 & Filho(a) do responsável e do cônjuge & NÃO ENTRA \\
				Bisneto(a) & 0 & 0 & Filho(a) somente do responsável & NÃO ENTRA \\
				Bisneto(a) & 0 & 0 & Enteado(a) & NÃO ENTRA \\
				Bisneto(a) & 0 & 0 & Genro ou nora & NÃO ENTRA \\
				Bisneto(a) & 0 & 0 & Pai, mãe, padrasto ou madrasta & NÃO ENTRA \\
				Bisneto(a) & 0 & 0 & Sogro(a) & NÃO ENTRA \\
				Bisneto(a) & 0 & 0 & Neto(a) & AMBIGUO \\
				Bisneto(a) & 0 & 0 & Bisneto(a) & NÃO ENTRA \\
				Bisneto(a) & 0 & 0 & Irmão ou irmã & NÃO ENTRA \\
				Bisneto(a) & 0 & 0 & Avô ou avó & NÃO ENTRA \\
				Bisneto(a) & 0 & 0 & Outros & AMBIGUO \\
				Irmão ou irmã & 1 & 0 & Irmão ou irmã & O PROPRIO \\
				Irmão ou irmã & 0 & 0 & Pessoa responsável pelo domicílio & NÃO ENTRA \\
				Irmão ou irmã & 0 & 0 & Conjuge & NÃO ENTRA \\
				Irmão ou irmã & 0 & 0 & Filho(a) do responsável e do cônjuge & NÃO ENTRA \\
				Irmão ou irmã & 0 & 0 & Filho(a) somente do responsável & NÃO ENTRA \\
				Irmão ou irmã & 0 & 0 & Enteado(a) & NÃO ENTRA \\
				Irmão ou irmã & 0 & 0 & Genro ou nora & NÃO ENTRA \\
				Irmão ou irmã & 0 & 0 & Pai, mãe, padrasto ou madrasta & PAI/MÃE \\
				Irmão ou irmã & 0 & 0 & Sogro(a) & NÃO ENTRA \\
				Irmão ou irmã & 0 & 0 & Neto(a) & NÃO ENTRA \\
				Irmão ou irmã & 0 & 0 & Bisneto(a) & NÃO ENTRA \\
				Irmão ou irmã & 0 & 0 & Irmão ou irmã & NÃO ENTRA \\
				Irmão ou irmã & 0 & 0 & Avô ou avó & NÃO ENTRA \\
				Irmão ou irmã & 0 & 0 & Outros & AMBIGUO \\
				Avô ou avó & 1 & 0 & Avô ou avó & O PROPRIO \\
				Avô ou avó & 0 & 0 & Pessoa responsável pelo domicílio & NÃO ENTRA \\
				Avô ou avó & 0 & 0 & Conjuge & NÃO ENTRA \\
				Avô ou avó & 0 & 0 & Filho(a) do responsável e do cônjuge & NÃO ENTRA \\
				Avô ou avó & 0 & 0 & Filho(a) somente do responsável & NÃO ENTRA \\
				Avô ou avó & 0 & 0 & Enteado(a) & NÃO ENTRA \\
				Avô ou avó & 0 & 0 & Genro ou nora & NÃO ENTRA \\
				Avô ou avó & 0 & 0 & Pai, mãe, padrasto ou madrasta & AMBIGUO \\
				Avô ou avó & 0 & 0 & Sogro(a) & NÃO ENTRA \\
				Avô ou avó & 0 & 0 & Neto(a) & NÃO ENTRA \\
				Avô ou avó & 0 & 0 & Bisneto(a) & NÃO ENTRA \\
				Avô ou avó & 0 & 0 & Irmão ou irmã & NÃO ENTRA \\
				Avô ou avó & 0 & 0 & Avô ou avó & AMBIGUO \\
				Avô ou avó & 0 & 0 & Outros & AMBIGUO \\
				Outros & 1 & 0 & Outros & O PROPRIO \\
				Outros & 0 & 0 & Pessoa responsável pelo domicílio & NÃO ENTRA \\
				Outros & 0 & 0 & Conjuge & AMBIGUO \\
				Outros & 0 & 0 & Filho(a) do responsável e do cônjuge & AMBIGUO \\
				Outros & 0 & 0 & Filho(a) somente do responsável & AMBIGUO \\
				Outros & 0 & 0 & Enteado(a) & AMBIGUO \\
				Outros & 0 & 0 & Genro ou nora & AMBIGUO \\
				Outros & 0 & 0 & Pai, mãe, padrasto ou madrasta & AMBIGUO \\
				Outros & 0 & 0 & Sogro(a) & AMBIGUO \\
				Outros & 0 & 0 & Neto(a) & AMBIGUO \\
				Outros & 0 & 0 & Bisneto(a) & AMBIGUO \\
				Outros & 0 & 0 & Irmão ou irmã & AMBIGUO \\
				Outros & 0 & 0 & Avô ou avó & AMBIGUO \\
				Outros & 0 & 0 & Outros & AMBIGUO \\ \bottomrule
\end{longtable}
	% ---
	
	
\end{anexosenv}


%---------------------------------------------------------------------
% INDICE REMISSIVO
%---------------------------------------------------------------------
\phantompart
\printindex
%---------------------------------------------------------------------

\end{document}
